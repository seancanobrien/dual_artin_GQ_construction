%! TEX root = ../../main.tex
We will work to re--frame the proof of \cref{thm:resteghini_proof} using the objects and tools developed in \cref{sec:masking_group_relations}.
To do so, let us first inspect exactly what kind of group elements are generated by the Hurwitz action, specifically, let us try to describe $B_n \codot h$ as a subset of $W$.

\begin{definition}
	For a group $G \cong \GroupPres{S \relations R}$ let $\ell_S \colon G \to \N$ be the usual length function with respect to the generating set $S$.
	We will omit $S$ when it is obvious from context.
\end{definition}
\begin{definition}
	For some group $G \cong \GroupPres{S \relations R}$ and $n \in \N$, we define the symbol $G|_n$ to be $\ell^{-1}(\Set{0, \ldots, n})$, i.e.~the group elements that have length less than or equal to $n$.
\end{definition}
\begin{example}
	Let $W$ be some rank--3 Coxeter group, let $h = (s_1,s_2,s_3)$ and let $B = B_3$.
	We can act on $h$ by generators of $B$ to get
	\begin{align*}
		\sigma_1 \cdot h      & = (s_2^{s_1}, s_1, s_3)                                                    \\
		\sigma_2 \cdot h      & = (s_1, s_3^{s_2}, s_2)                                                    \\
		\sigma_1^{-1} \cdot h & =  (s_2, s_1^{\left(s_2^{-1}\right)}, s_3) & ( & = (s_2, s_1^{s_2}, s_3) ) \\
		\sigma_2^{-1} \cdot h & = (s_1, s_3, s_2^{\left(s_3^{-1}\right)})  & ( & = (s_1, s_3, s_2^{s_3}))  \\
	\end{align*}
	We see that $B|_1 \codot h$ is a subset of $S \cup \Set{s_i^{s_j} \given s_i,s_j \in S, s_i \neq s_j} $.
	\label{ex:rank_3_hurwitz_action}
\end{example}

\begin{definition}
	Let $(W,S)$ and $h = (s_1, \ldots, s_n)$ be some Coxeter system and Coxeter element
	We will recursively define a sequence of sets
	Let $C_0 \coloneqq S$
	Define  $C_{i+1}$ to be
	\[
		C_{i+1} \coloneqq C_i \cup \Set{a^{b} \given a,b \in C_i, a \neq b}
		.\]
\end{definition}
\begin{definition}
	We call an element of $C_i \setminus (C_{i-1} \cup \cdots \cup C_0)$ an element of conjugation length $i$.
	We can use a compact symbol, which is best described recursively, to describe such an element of conjugation length $i$.
	Setting $[s_0] \coloneqq s_0$, we define
	\[
		[s_0, \ldots, s_i] \coloneqq s_i^{[s_0, \ldots, s_{i-1}]}
		.\]
\end{definition}

\begin{remark}
	$[x_0, \ldots, x_i] ^ {[y_0, \ldots , y_j]} = [x_0, \ldots, x_i, y_0, \ldots, y_j]$
\end{remark}

This notation allows for a compact description of the Hurwitz action, where conjugation becomes concatenation.
We can take this further by writing $i$ instead of  $s_i$.
See \href{https://mathoverflow.net/questions/481777/tuple-rearrangement-a-combinatoric-problem-emerging-from-the-hurwitz-action-on}{this} MathOverflow post I made, asking if anyone knows what this game is.

\begin{example}
	Let us continue as in \cref{ex:rank_3_hurwitz_action}.
	We have
	\begin{align*}
		\sigma_1 \cdot h         & = ([2,1],[1],[3])     \\
		\sigma_2 \cdot h         & = ([1],[3,2],[2])     \\
		\sigma_1^{-1} \cdot h    & = ([2],[1,2],[3])     \\
		\sigma_2^{-1} \cdot h    & = ([1],[3],[2,3])     \\
		\sigma_1^2 \cdot h       & = ([1,2,1],[2,1],[3]) \\
		\sigma_1\sigma_2 \cdot h & =  ([3,2,1],[1],[2])  \\
	\end{align*}

\end{example}

\begin{theorem}
	$C_i = \Set{[x_0, \ldots, x_j] \given  j \leq i , \text{all} \; x_k \in S} $
\end{theorem}

\begin{theorem}
	Given some Coxeter system $(W, S)$ and Coxeter element $h$, we have $(B_n)|_j \codot h \subseteq C_j$.
	Thus, for each  $t \in T \cap B_n \cdot h$, there exists a $j \in N$ such that  $t \in C_j$
\end{theorem}

\begin{remark}
	These $C_i$ grow in size much faster than $(B_n)|_i \cdot h$.
	Also, there will typically exist $[x_0, \ldots, x_j] \notin T_0$.
	A--priori, the exact elements that appear in $T_0$, depend on combinatorics specific to $h$.
\end{remark}

\begin{theorem}
	By selecting the appropriate strings that do occur
\end{theorem}

\begin{definition}
	Let $C_i_*$ be defined just like  $C_i$, except its elements are words, rather than group elements
	We do not perform any multiplication.
	Similarly define $[$
\end{definition}

