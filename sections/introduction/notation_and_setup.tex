%! TEX root = ../../main.tex
We will begin by setting up our notation for Coxeter and Artin groups.

Let $W$ and  $A$ respectively be the Coxeter and Artin groups associated to some Coxeter system  $\Gamma$.
In this document, the specific Coxeter system is arbitrary and constant.

Our Coxeter and Artin groups have a conventional finite generating set which we associate to the set $S$.
In this document, $n$ will always denote $\Abs{S}$, the rank of our Coxeter system.
Rather than $S$ being a subset of  $W$ or  $A$, we will consider $S$ to be a free generating set, with $F_S$ the free group on $S$.
For any quotient $G$  of $F_S$ (particularly  $W$ or $A$), we denote the quotient surjection  $\pi_{(S,G)} \colon F_S \to G$.
Where $S$ is obvious from context, we may drop it from this notation in favour of  $\pi_G \colon F_S \to G$.
For shorthand, we will denote  $\pi_{W}(S)$ and  $\pi_{A}(S)$ by $S_W$ and  $S_A$ respectively.

To specify a dual Artin group, we must in principle choose a Coxeter element in $W$.
We will always denote this choice  $w$.
We will denote the generating simple reflections $W$ by $S$.
The elements of $S$ will always be denoted $s_i$ for  $1\leq i \leq n$ such that $w$ is a lexicographic product, i.e.~$w=\pi_{W}(s_1s_2\cdots s_n)$.
For any object (set or tuple) $X$ with such an indexing $x_i \in X$, and function $f$, we may use the notation $\Set{f(x_i)}$ as shorthand for $\Set{f(x_i) \given x_i \in X}$.

% Let $R \coloneq (S_W)^{W} \subseteq W$, $T \coloneq (S_A)^{A} \subseteq A$ and $U \coloneq S^{F_S}$ denote all conjugates of generators in $W$, $A$ and $F_S$ respectively.
Let $R \coloneq (S_W)^{W} \subseteq W$ denote all conjugates of generators in $W$.
We call $R$ the \emph{reflections} of  $W$.

