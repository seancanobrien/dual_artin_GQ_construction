\subsection{Some facts about the objects involved}
Here we will state some facts and try to point to the relevant place to read up on them.

Throughout, let $W = (W,S)$ be some Coxeter system.
Let $\Set\{s_i\}_{1 \leq i \leq n}$ be some ordering on $S$ and let $h = s_1s_2\cdots s_n$ be a choice of Coxeter element resulting from the ordering on $S$.
From here forward, fix $W = (W,S)$ and $h$.
Let $T = S^W$ be the set of reflections in $W$. This is all conjugations of $S$ in $W$.
We define the dual Artin group $A^*  = A^*(W,S,h)$, as in \cite{resteghini_free_2024} or \cite{paolini_salvetti_kpi1_2021}.

Let $B_n$ denote the braid group on $n$ strands generated by $\sigma_1, \sigma_2, \ldots \sigma_{n-1}$ in the usual way.
Let $x^y = y x y^{-1}$ denote conjugation in groups.

\begin{definition}
	Define an action $\cdot$ of $B_n$ on $T^n$ such that
	\[
		\sigma_i \cdot  \left( t_1, \ldots t_n \right) = \left( t_1, \ldots t_{i-1}, t_{i+1}^{t_i}, t_{i}, t_{i+2} \ldots t_n \right)
	\]
	\label{def:braid_action}
\end{definition}
\begin{theorem}
	We can map $T^n$ to $W$ by multiplying all the elements in the tuple in order.
	Considering this map, the action of $B_n$ is transitive on minimal $T$--factorisations of $h$.
\end{theorem}
Let $T \cap (B_n \cdot h)$ denote all reflections that occur in any position in a minimal  $T$--factorisation of $h$.
\begin{theorem}
	For each $ r \in T \cap (B_n \cdot h)$, and for each $i \in \Set{1, \ldots n} $ there exists a minimal $T$--factorisation $(t_1, \ldots t_n) \in (B_n \cdot h)$ such that $t_i = r$. So every reflection that appears, appears in every possible position.
\end{theorem}
\begin{theorem}
	The above theorem also applies to subwords. For each $(r_1, \ldots r_n) \in (B_n \cdot h)$, each $i \leq n$ and each  $j \leq n-i$, there exists a $(t_1, \ldots t_n) \in (B_n \cdot h)$ such that $(t_j, t_{j+1}, \ldots, t_{j+i}) = (r_1, \ldots, r_i)$.
\end{theorem}

\begin{theorem}[\hspace{1sp}{\cite[Proposition 10.1]{mccammond_sulway_artin_2017}}]
	The generating $W$ generating set $S$ is contained in $T \cap (B_n \cdot h)$ and this is also a generating set for $A^*$.
	The natural inclusion of $\overline{S} \subseteq A$ in to $A^*$ induces a surjective homomorphism.
	\label{thm:homo_art_to_dual_art}
\end{theorem}

\begin{theorem}[\hspace{1sp} {\cite[Lemma 7.11]{bessis_topology_2004}} ]
	Let $[t]$ denote an abstract generator coming from some $t \in T \cap(B_n \cdot h)$. The dual Artin group has the following presentation.
	\[
		A^*(W,S,h) \cong \GroupPres{T \cap (B_n \cdot h) \relations \Set{ [t t^\prime t][t][t^\prime]^{-1}[t]^{-1} \given (t,t^\prime,x_3, \ldots x_n) \in B_n \cdot h}}
		.\]
	\label{thm:dual_braid_relations}
\end{theorem}

We call such $[t t^\prime t][t][t^\prime]^{-1}[t]^{-1}$ \emph{dual braid relations}.

The following is a basic group theoretic fact, but it will be useful to explicitly state and prove it. For some set $S$, let $F_S$ be the free group generated by $S$.
\begin{lemma}
	Suppose we have the groups $G_1$ and $G_2$ such that $G_1$ has a presentation $G_1 \cong \GroupPres{S \relations R}$ where $R$ is a set of words in $S \cup S^{-1}$ that are the identity in $G_1$. Further suppose we are given some map $f \colon S \to G_2$. For all $s \in S$, define $f(s^{-1})$ to be $f(s)^{-1}$. If for each word $s_1s_2 \cdots s_n \in R$ we have that $f(s_1)f(s_1)\cdots f(s_n) = 1$ in $G_2$ then we can extend $f$ to a homomorphism  $h \colon G_1 \to G_2$ where $h|_{S} = f$ considering $S$ as a subset of $G_1$. \label{lem:extend_map_to_homomorphism}
\end{lemma}
\begin{proof}
	We can define a homomorphism $f^\prime \colon F_{S} \to G_2$ recursively by setting $f^\prime(1) =1$, $f^\prime(s^{\pm 1}) = f(s)^{\pm 1}$ for all $s \in S$ and then setting $f^\prime(s_1s_2 \cdots s_n) = f^\prime(s_1)f^\prime(s_2\cdots s_n)$ for all words $s_1s_2\cdots s_n \in \Set{S \cup S^{-1}}^*$. Let $N$ be the minimal normal subgroup in $F_{S}$ which contains the elements of $R$ such that $G_1 \cong F_{S}/N$. By assumption, for each $r \in R$ we have $f^\prime(r) = 1$. So $R \subseteq \ker(f^\prime)$. So $N \subseteq \ker(f^\prime)$. By the universal property of the quotient we have the following commutative diagram where the unique map $h$ is the required homomorphism.
	\begin{equation*}
		\begin{tikzcd}
			F_{S} \ar[r, "f^\prime"] \ar[d, "q"'] &G_2 \\
			F_{S}/N \cong G_1 \ar[ru, "\exists ! h"']
		\end{tikzcd}
	\end{equation*}
\end{proof}
Note that if $\GroupPres{\image(f)} = G_2$ then $h$ is a surjection.


