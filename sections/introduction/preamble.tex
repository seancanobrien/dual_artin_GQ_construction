%! TEX root = ../../main.tex
We will begin by setting up our notation for Coxeter and Artin groups.

Let $W$ and  $A$ respectively be the Coxeter and Artin groups associated to some Coxeter system  $\Gamma$.
In this document, the specific Coxeter system is arbitrary and constant.

Our Coxeter and Artin groups have a conventional finite generating set $S$.
In this document, $n$ will always denote $\Abs{S}$, the rank of our Coxeter system.
We will consider $S$ to be a free generating set, with $F_S$ the free group on $S$.
For any quotient $G$  of $F_S$ (particularly  $W$ or $A$), we denote the quotient surjection  $\pi_{(S,G)} \colon F_S \to G$.
Where $S$ is obvious from context, we may drop it from this notation in favour of  $\pi_G \colon F_S \to G$.
For shorthand, we will denote  $\pi_{W}(S)$ and  $\pi_{A}(S)$ by $S_W$ and  $S_A$ respectively.

To specify a dual Artin group, we must in principle choose a Coxeter element in $W$.
We will always denote this choice  $w$.
We will denote the generating simple reflections $W$ by $S$.
The elements of $S$ are indexed by the naturals from 1 to $n$ such that $w$ is a lexicographic product, i.e.~$w=\pi_{W}(s_1s_2\cdots s_n)$.

Let $q \colon A \to W$ denote the natural surjection associated to these groups.
For each $s \in S$, we denote the \emph{positive lift} of $s$ by  $\overline{s} \in A$.
This is a section of  $q$ restricted to the generating set of $A$ such that $\overline{q(\alpha)} = \alpha$ for each  $\alpha$ in the (positive) generating set of  $A$.
Accordingly, we may identify the generating set of  $A$ with  $\overline{S}=\Set{\overline{s} \given s \in S}$.
It is tempting to extend $s \mapsto \overline{s}$ to a  map $W \to A$, but this would only be well-defined on words in $S$, not elements of $W$.

Let $R \coloneq (S_W)^{W} \subseteq W$, $T \coloneq (S_A)^{A} \subseteq A$ and $U \coloneq S^{F_S}$ denote all conjugates of generators in $W$, $A$ and $F_S$ respectively.
We call $R$ the \emph{reflections} of  $W$.

Let  $B_n$ denote the braid group on  $n$ strands with standard generating set $\Set{\sigma_i}_{1 \leq i < n} $.
Given any group $G$, there is an action of  $B_n$ on  $G^n$ where
\[
	\sigma_i \cdot  \left( g_1, \ldots g_n \right) = \left( g_1,\ldots,g_{i-1}, g_{i+1}^{\left(g_i^{-1}\right)}, g_{i}, g_{i+2}, \ldots, g_n \right)
	.\]
This is called the \emph{Hurwitz action}.
This action preserves the index-wise product $(g_1, \ldots, g_n) \mapsto g_1g_2\cdots g_n$.
Given some $X \subseteq G$, if $X^G = G$, then the Hurwitz action is also well-defined on $X^n$, and this makes any such $X$ a  $B_n$-set.
In particular, we wish to consider this action on $S^n$, $T^n$ and $U^n$.


Let $p_1 \colon G^n \to G$ denote generic projection to the first coordinate of a tuple.
We are interested in the tuples that occur due to the Hurwitz action.
The set $B_n \cdot \tau$ is the set of all such tuples for some tuple  $\tau$.
The set $\hurref(\tau)$ is the set of all  $g \in G$ that occur in any position in  $B_n \cdot \tau$.
By choosing braids that move elements to the first position in the tuple, we see that $\hurref(\tau) = p_1(B_n \cdot \tau)$.

Associated to a tuple, there is a monoid construction, which we will call the \emph{Hurwitz monoid}.
\begin{definition}
	Let $G$ be some group, and let $\tau$ be some tuple  in $G^k$.
	Let $X$ denote  $\hurref(\tau)$.
	Suppose there in there is a tuple in  $B_k \cdot \tau$ that begins $r_1,r_2$.
	Let $r_3$ denote the element of $X$ equal in $G$ to  $r_1^{r_2^{-1}}$.
	We call any such $(r_1,r_2,r_3)$ a \emph{Hurwitz triple}.
	In this context, we define the \emph{Hurwitz monoid} $I$ associated to $\tau$ as follows.
	Decorate elements $x \in X$ as $[x]$ to distinguish them as generators of $I$ rather than elements of $G$.
	\[
		I \coloneq \GroupPres{[X] \relations \Set{ [r_1][r_2] = [r_3][r_1] \given (r_1,r_2,r_3) \text{ is a Hurwitz triple}}}
		.\]
\end{definition}
We call the corresponding group, the \emph{Hurwitz group}.
\begin{lemma}
	Given a group $G$ and a tuple $\tau \in G^k$, the Hurwitz group $H$ associated to $\tau$ is generated by the elements of $\tau$.
\end{lemma}
\begin{proof}
	Let $\ell \colon B_n \to \N$ denote minimum word length with respect to the standard generating set for $B_n$.
	Let $x \in \hurref(\tau) \setminus \Set{\tau_i}$.
	Let $\beta \in B_n$ be the minimum length braid such that $\beta \cdot \tau =(\mu,\cdots,\mu_n)$ and $x=\mu_i$ for some $i$.
	We will show that such an $[x]$ can be written in terms of generators appearing in $\gamma \cdot \tau$ where  $\gamma$ is a braid with $\ell(\gamma)<\ell(\beta)$.
	Induction then completes the proof.

	Since $x \notin \Set{\tau_j}$, we know $\ell(\beta)>0$.
	Recall our notation for the set of standard generators for $B_n$ is $\Set{\sigma_j}$.
	Since $\beta$ is a braid of minimum length with respect to these generators, we know that there is a factorisation of $\beta$ in the $\Set{\sigma_j} $ such that the last factor is either $\sigma_i$ or  $\sigma_{i-1}^{-1}$, since these are the only generators whose action puts something new in the $i^{\text{th}}$ position of the tuple.

	Suppose the last factor of $\beta$ is $\sigma_i$.
	Let $(\nu_1,\ldots,\nu_n) = \beta\sigma_i^{-1} \cdot \tau$, so $[x] = [\mu_i] = [\nu_{i+1}\nu_i\nu_{i+1}^{-1}]$.
	Since $\ell(\beta \sigma_i^{-1}) \leq \ell(\beta)-1$, if we can express $x$ in terms of the set $\Set{[\nu_j]}$ then we would be done.
	Since we can bring $\nu_i$ and  $\nu_{i+1}$ to the start of the tuple, we have the equation $[x]=[\nu_{i+1}][\nu_i][\nu_{i+1}]^{-1}$ in $H$.

	We make a similar argument if the last factor of $\beta$ is  $\sigma_{i-1}^{-1}$.
\end{proof}
The following is not a standard definition.
\begin{definition}
	Given a Coxeter element $w \in W$, the \emph{dual Artin group} associated to  $w$ is defined to be the Hurwitz group associated to any $(r_1,\ldots,r_n)$ such that all $r_i \in R$ and  $r_1r_2\cdots r_n=w$.
\end{definition}
The obvious tuple to consider is $(\pi_W(s_1),\ldots,\pi_W(s_n))$.
Since our choice of $w$ is constant, we will denote our dual Artin group  $A^\vee$ (omitting any mention of  $w$).

Let $a,b \in G$ for some group  $G$.
Let $\Pi(a,b;k)$ denote the alternating product of $a$ and  $b$, beginning with $a$, and of length  $k$, where  $a$ and  $b$ are any group elements.
For example  $\Pi(a,b;5)=ababa$.
\begin{lemma}
	Let $G$ be a group such that there exists a homomorphism $\phi \colon A \to G$ from our Artin group.
	Consider the tuple $\tau = (\phi \circ \pi_A)^n(s_1,\ldots,s_n) \in G^n$.
	There is a natural surjection from $A$ to $H$, the Hurwitz group associated to $\tau$.
\end{lemma}
\begin{proof}
	For each $i$, let $t_i$ denote  $\pi_A(s_i)$ and  $u_i$ denote  $\phi(t_i)$.
	We will show that the map $t_i \mapsto u_i$ extends to a homomorphism.
	To do so, we need to show that any Artin-like equation of the form $\Pi(u_i,u_j;k) = \Pi(u_j,u_i;k)$ associated to a defining relation in $A$ also holds in $H$.

	For any $i < j$, there is a braid $\beta$ such that  $\beta \cdot (u_1,\ldots,u_n)$ begins with $u_i,u_j$.
	Thus, without loss of generality, we can assume for the following argument that $i=1$ and  $j=2$
	Let us examine what happens when we keep acting by $\sigma_1$ on  $(u_1,\ldots,u_n)$.

	Acting once we get the relation
	\[
		[u_2u_1u_2^{-1}]=[u_2][u_1][u_2]^{-1}
		.\]
	Once more, with some substitution using the above
	\[
		[u_1u_2u_1u_2^{-1}u_1^{-1}] = [u_1][u_2u_1u_2^{-1}][u_1]^{-1} = [u_1][u_2][u_1][u_2]^{-1}[u_1]^{-1}
		.\]
\end{proof}
So the index-wise product of  $\pi^n_W(\tau)$ is  $w$.
In this context, $\pi^n_G$ is a  $B_n$-set morphism from  $F_S^n$ to  $G^n$, where $G$ is either  $W$ or  $A$.
It is true that $\pi_W^n(B_n \cdot \tau)= B_n \cdot \pi^n_W(\tau)$ gives every minimal factorisation of  $w$ in to elements of $R$.

