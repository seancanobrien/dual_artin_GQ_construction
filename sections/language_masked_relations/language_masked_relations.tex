%! TEX root = ../../main.tex
Let $F \colon \cset \to \cgrp$ be the free functor and denote its action on objects as $F_S \coloneqq F(S)$ and its action on morphisms as  $f^* \coloneqq F(f \colon S \to T)$. To each quotient $G$ of $F_S$ we associate the surjective homomorphism $\pi_{(S,G)} \colon F_S \to G$ which is projection. In this context, we can frame a basic group theoretic fact.

\begin{theorem}
	Suppose we have two groups $G_1$ and $G_2$. Suppose that $G_1 \cong \GroupPres{S \relations R}$ where $R$ is a subset of $F_S$ for which $\pi_{(S,G_1)}(R) = \Set{1}$.
	Given a map $f \colon S \to G_2$, if  $\pi_{(G_2, G_2)} \circ f^*(R) = \Set{1} \subseteq G_2$, then $f$ defines a homomorphism $h \colon G_2 \to G_2$.
	\label{thm:homomorphism_universal_prop}
\end{theorem}
\begin{proof}
	Let $N$ be the minimal normal subgroup in $F_{S}$ which contains the elements of $R$ such that $G_1 \cong F_{S}/N$.
	By assumption, $R \subseteq \ker(\pi_{(G_2,G_2)} \circ f^*)$, which is normal, thus $N \subseteq \ker(\pi_{(G_2,G_2)} \circ f^*)$.
	By the universal property of the quotient we have the following commutative diagram where the unique map $h$ is the required homomorphism.
	\begin{equation*}
		\begin{tikzcd}
			F_{S} \ar[rr, "\pi_{(G_2,G_2)} \circ f^*"] \ar[d, "\pi_{(S,G_1)}"'] & & G_2 \\
			F_{S}/N \cong G_1 \ar[rru, "\exists ! h"']
		\end{tikzcd}
	\end{equation*}
\end{proof}

The above theorem defines a homomorphism $h \colon \GroupPres{S \relations R} \cong G_1 \to G_2$ when a map $f \colon S \to G_2$ is compatible with the relations in $G_1$ and $G_2$.
If we also had knowledge of a generating set for $G_2$, we could construct homomorphisms in a different way.
\begin{theorem}
Suppose $G_1 \cong \GroupPres{S_1 \relations R_1}$ and $G_2 \cong \GroupPres{S_2}$.
Given a map $f \colon S_1 \to S_2$, if there exists a map  $h$ such that the following diagram commutes, then $h$ is a homomorphism.
\[
\begin{tikzcd}
	S_1 \ar[d, "f"] \ar[r, hook, "i_1"] & F_{S_1} \ar[d, "f^*"] \ar[rr, "\pi_{(S_1, G_1)}"] & & \ar[d, color=red, "h"] G_1 \\
	S_2 \ar[r, hook, "i_2"] & F_{S_2} \ar[rr, "\pi_{(S_2, G_2)}"] & & G_2
\end{tikzcd}
\]
\end{theorem}
\begin{proof}
	Apply \cref{thm:homomorphism_universal_prop} to the map $\pi_{(S_2, G_2)} \circ i_2 \circ f$.
\end{proof}

Consider the above diagram, but replacing $F_{S_1}$ with some subset $Q \subseteq F_{S_1}$. We will now construct a group, which we will denote $G^Q$, where such a diagram defines a homomorphism from $G^Q$ to $G_2$.

 \begin{definition}
	 Suppose we have a group $G \cong \GroupPres{S}$.
	 Fix some $Q \subseteq F_S$ such that $S \subseteq Q$.
	 Let $\pi \coloneq \pi_{(S,G)}$.
	 We have the following maps.
	 \[
	 \begin{tikzcd}
		 S \ar[r, hook, "i"] & Q \ar[r, "\pi"] & \pi(Q)
	 \end{tikzcd}
	 .\]
	 Define the \emph{group with relations visible in $Q$}, $G^Q$ to be  
	 \[
		 G^Q \coloneqq \GroupPres{\pi(Q) \relations \Set{\pi(q) = (\pi \circ i)^*(q) \given q \in Q} }
	 .\]
	 \label{def:G_Q}
\end{definition}
\begin{remark}
	Do not think of  the generators $\pi(Q)$ as elements of $G$.
	They are abstract generators.
	Specifically,  $G^Q$ is a quotient of  $F_{\pi(Q)}$.
	Thus, our relations should be equations in $F_{\pi(Q)}$, which they are.
\end{remark}
\begin{lemma}
	Let $G \cong \GroupPres{S}$ be a group and consider the following setup.
	\[
	\begin{tikzcd}
		S \ar[r, hook, "i_S"] & F_S \ar[rr, "\pi_{(S,G)}"] & & G \ar[r, hook, "i_G"] & F_G \ar[rr, "\pi_{(G,G)}"] & & G
	\end{tikzcd}
	\]
	The following maps are equal.
	\[
		\pi_{(G,G)} \circ i_G \circ \pi_{(S,G)} = \pi_{(G,G)} \circ (\pi_{(S,G)} \circ i_S)^*
	.\]
	\label{lem:group_projections}
\end{lemma}
\begin{theorem}
	Suppose we have two groups $G_1 \cong \GroupPres{S_1}$ and $G_2 \cong \GroupPres{S_2}$.
	Fix some $Q \subseteq F_{S_1}$ such that  $S \subseteq Q$.
	Let $\overline{Q} \coloneqq \pi_{(S_1, G_1)}(Q) $.
	If in the following diagram, there exists a map $f$ that makes the diagram commute, then there is a homomorphism $h \colon G^Q \to G_2$.
	\[
		\begin{tikzcd}
			S_1 \ar[d, "\theta"] \ar[r, hook, "i_{S_1}"] & Q \ar[d, "\theta^*"] \ar[rr, "\pi_{(S_1,G_1)}"] & & \overline{Q} \ar[d, color=red, "\exists f ?"] \\
			S_2 \ar[r, hook, "i_{S_2}"] & F_{S_2} \ar[rr, "\pi_{(S_2,G_2)}"] & & G_2 
		\end{tikzcd}
		\]
\end{theorem}
\begin{proof}
	Suppose we have such a map $f$.
	Note that  $\overline{Q}$ is a generating set for  $G^Q$, thus we have the setup of \cref{thm:homomorphism_universal_prop}.
	We use the following commutative diagram to define some inclusion maps and as a reference for the setup.
	\begin{equation}
		\begin{tikzcd}
			S_1 \ar[d, "\theta"] \ar[r, hook, "i_{S_1}"] & Q \ar[d, "\theta^*"] \ar[rr, "\pi_{(S_1, G_1)}"] & & \overline{Q} \ar[d, "f"] \ar[r, "i_{\overline{Q}}"] & F_{\overline{Q}} \ar[d, "f^*"]
			\\ S_2 \ar[r, hook, "i_{S_2}"] & F_{S_2} \ar[rr, "\pi_{(S_2, G_2)}"] & & G_2 \ar[r, hook, "i_{G_2}"] & F_{G_2} \ar[rr, "\pi_{(G_2, G_2)}"] & & G_2
		\end{tikzcd}
		\label{eqn:G_Q_proof_setup}
	\end{equation}
	We can construct a homomorphism if $\pi_{(G_2, G_2)} \circ f^*(R) = \Set{1}$, where $R$ are the relations for $G^Q$, defined in \cref{def:G_Q}.
	For some $q \in Q$, the corresponding element in $R$ equalling the identity is  $i_{\overline{Q}} \circ \pi_{(S_1,G_1)}(q)(\pi_{(S_1,G_1)} \circ i_1)^*(q)^{-1}$.
	Using that $\pi_{(G_2,G_2)} \circ f^*$ is a homomorphism, we have
	\begin{align*}
		\pi_{(G_2,G_2)} \circ f^* \left(i_{\overline{Q}} \circ \pi_{(S_1,G_1)}(q)(\pi_{(S_1,G_1)} \circ i_1)^*(q)^{-1}\right) =
		\\ \left(\pi_{(G_2,G_2)} \circ f^* \circ i_{\overline{Q}} \circ \pi_{(S_1,G_1)} (q)\right) \left( \pi_{(G_2,G_2)} \circ f^* \circ\left( \pi_{(S_1,G_1)} \circ i_{S_1}\right)^*(q)^{-1} \right).
	\end{align*}
	We will concentrate on each factor separately.

	First we consider the first factor.
	Since $\pi_{(S_1, G_1)}(q) \in \overline{Q}$, by the rightmost commuting square of \eqref{eqn:G_Q_proof_setup}, we have $f^* \circ i_{\overline{Q}} \circ \pi_{(S_1, G_1)}(q) = i_{G_2} \circ f \circ \pi_{(S_1,G_1)} (q)$.
	This gives us
	\begin{align*}
		\pi_{(G_2,G_2)} \circ f^* \circ i_{\overline{Q}} \circ \pi_{(S_1,G_1)} (q) = \pi_{(G_2,G_2)} \circ i_{G_2} \circ f \circ \pi_{(S_1,G_1)}(q).
	\end{align*}
	We then use the middle commuting square of \eqref{eqn:G_Q_proof_setup} to give us
	\begin{align*}
		\pi_{(G_2,G_2)} \circ i_{G_2} \circ f \circ \pi_{(S_1,G_1)}(q) = \pi_{(G_2,G_2)} \circ i_{G_2} \circ \pi_{(S_2,G_2)} \circ \theta^*(q).
	\end{align*}
	We then use \cref{lem:group_projections} and functoriality, giving us
	\begin{align*}
		\pi_{(G_2,G_2)} \circ i_{G_2} \circ \pi_{(S_2,G_2)} \circ \theta^*(q) &= \pi_{(G_2,G_2)} \circ \left( \pi_{(S_2,G_2)} \circ i_{S_2} \right)^*  \circ \theta^*(q)
										   \\ &= \pi_{(G_2,G_2)} \circ \left( \pi_{(S_2,G_2)} \circ i_{S_2} \circ \theta \right)^*(q)
	\end{align*}

	We now concentrate on the second factor. Using functoriality, then the middle commuting square of \eqref{eqn:G_Q_proof_setup}, we get
	\begin{align*}
		\pi_{(G_2,G_2)} \circ f^* \circ\left( \pi_{(S_1,G_1)} \circ i_{S_1}\right)^*(q)^{-1} &= \pi_{(G_2,G_2)} \circ \left( f \circ \pi_{(S_1,G_1)} \circ i_{S_1}\right)^*(q)^{-1}
												  \\ &= \pi_{(G_2,G_2)} \circ \left(\pi_{(S_2,G_2)} \circ \theta^* \circ i_{S_1}\right)^*(q)^{-1}.
	\end{align*}
	Then, we use the leftmost commuting square of \eqref{eqn:G_Q_proof_setup} to get
	\begin{align*}
		\pi_{(G_2,G_2)} \circ \left(\pi_{(S_2,G_2)} \circ \theta^* \circ i_{S_1}\right)^*(q)^{-1} = \pi_{(G_2,G_2)} \circ \left(\pi_{(S_2,G_2)} \circ i_{S_2} \circ \theta \right)^*(q)^{-1}.
	\end{align*}

	This is the inverse of the left factor.
	\end{proof}
To show that 
we could define a ma
For a set of symbols $\mathcal{S}$, define  $\mathcal{S}^*$ to be the language of all words in  $\mathcal{S}$.
Consider the following and compare with \cref{lem:extend_map_to_homomorphism}.

\begin{lemma}
	Let $G$ and $G^\prime$ be groups generated by $S$ and $S^\prime$ respectively.
	Let $\pi_G \colon (S \cup S^{-1})^* \to G$ denote multiplication of words in the group, and similarly for $\pi_{G^\prime}$.
	Suppose we have a map  $\theta \colon S \to S^\prime$.
	We can extend $\theta$ to a function $\theta^* \colon (S \cup S^{-1})^* \to (S^\prime \cup (S^\prime)^{-1})^*$ in the obvious way.
	Consider the following diagram.
	\[
		\begin{tikzcd}
			S \ar[d, "\theta"] \ar[r, hook] & (S \cup S^{-1})^* \ar[d, "\theta^*"] \ar[r, "\pi_G"] & G \\
			S^\prime \ar[r, hook] & (S^\prime \cup (S^\prime)^{-1})^* \ar[r, "\pi_{G^\prime}"] & G^\prime
		\end{tikzcd}
		.\]
	The left square commutes.
	If there exists a \emph{map} $f \colon G \to G^\prime$ that makes the right square commute, then $f$ is a homomorphism.
	\label{lem:homomorphism_squares}
\end{lemma}
\begin{remark}
	Note that if $G \cong \GroupPres{S \relations R}$ then such a map (and thus homomorphism) $f$ exists exactly when  $\pi_{G^\prime} \circ \theta^* (R) = 1$.
	The above lemma frames up the construction in \cref{lem:extend_map_to_homomorphism}, but does not provide any useful way to detect when such a homomorphism is possible.
	\label{rem:when_does_map_exist}
\end{remark}

\begin{definition}[Group with relations visibile in Q]
	Let $G \cong \GroupPres{S \relations R}$.
	Let  $Q$ be some language in  $S \cup S^{-1}$ such that, considering  $S$ as one-letter words, $S \subseteq Q$.
	We define the group with relations visible in $Q$,  $G^Q$ to be
	\[
		G^Q \coloneqq \GroupPres{\pi_G(Q) \relations \Set{\pi_G(q_1q_2\cdots q_n) = \pi_G(q_1)\pi_G(q_2) \cdots \pi_G(q_n) \given q_1\cdots q_n \in Q} }
		.\]
	Note that the generators $\pi_G(Q)$ are abstract generators.
	They do not necessarily inherit anything from $G$.
	The only equations that are necessarily true in  $G^Q$ (which are also tautologically true in $G$) are those spelled out by the language $Q$.
	\label{def:group_relations_visible_in_Q}
\end{definition}

\begin{remark}
	$G^Q$ is generated by the generators $\pi_G(S)$.
	$G$ is a quotient of  $G^Q$.
\end{remark}

\begin{theorem}
	Let $G \cong \GroupPres{S \relations R}$, let  $G^\prime$ be generated by $S^\prime$ and let $Q$ be a language in  $S \cup S^{-1}$ such that $S \subseteq Q$ as in \cref{def:group_relations_visible_in_Q}.
	Suppose we have a map $\theta \colon S \to S^\prime$ and define $\theta^*$ as in \cref{lem:homomorphism_squares}.
	For brevity, let $\overline{Q} \coloneqq \pi_G(Q)$.
	Consider the following diagram.
	\[
		\begin{tikzcd}
			S \ar[d, "\theta"] \ar[r, hook] & Q \ar[d, "\theta^*"] \ar[r, "\pi_G"] & \overline{Q} \ar[d, color=red, "\exists f ?"] \ar[r, hook] & (\overline{Q} \cup \overline{Q}^{-1})^* \ar[d, "\pi_{G^Q}"]\\
			S^\prime \ar[r, hook] & (S^\prime \cup (S^\prime)^{-1})^* \ar[r, "\pi_{G^\prime}"] & G^\prime & G^Q \ar[l, color=red, "\implies \exists g"']
		\end{tikzcd}
		.\]
	The left square commutes.

	Finally, the theorem is as follows.
	If there exists a map $f \colon \pi_G(Q) \to G^\prime$ that makes the middle square commute, then there is a homomorphism $g \colon G^Q \to G^\prime$ such that the right square commutes.
	\label{thm:commuting_diagram_GQ}
\end{theorem}
\begin{proof}
	Suppose we have such a diagram and map $f$.
	Note that, in $f$, we have a map from the generating set of $G^Q$ to $G^\prime$, which can be considered a generating set for $G^\prime$.
	Accordingly, replace $\theta$ with  $f$ in \cref{lem:homomorphism_squares}.
	\[
		\begin{tikzcd}
			\overline{Q} \ar[d, "f"] \ar[r, hook] & (\overline{Q} \cup \overline{Q}^{-1})^* \ar[d, "f^*"] \ar[r, "\pi_{G^Q}"] & G^Q \\
			G^\prime \ar[r, hook] & (G^\prime \cup (G^\prime)^{-1})^* \ar[r, "\pi_{G^\prime}"] & G^\prime
		\end{tikzcd}
		.\]
	If we can show we have a map $h \colon G^Q \to G^\prime$ such that the right square commutes, then by \cref{lem:homomorphism_squares} this is a homomorphism.
	This will also satisfy the necessary commutation relations.

	We can use the defining presentation for $G^Q \cong \GroupPres{\pi_G(Q) \relations R}$ from \cref{def:group_relations_visible_in_Q} and \cref{rem:when_does_map_exist} to show that $g$ exists.
	Let $R$ be our relations from \cref{def:group_relations_visible_in_Q}.
	Showing $\pi_{G^\prime} \circ f^*\left( R \right) = 1 $ will finish our proof.

	Let $(q_1q_2\cdots q_n) = q \in Q$.
	This corresponds to the relation $\pi_G(q_1\cdots q_n) = \pi_G(q_1)\cdots \pi_G(q_n) $ in $G^Q$.
	Each of $\pi_G(q)$ and the  $\pi_G(q_i)$ are in  $\pi_G(Q)$, so  $f^*(\pi_G(q)) = f(\pi_G(q))$ and  $f^*(\pi_G(q_i)) = f(\pi_G(q_i))$.
	By our assumed commuting diagram,  $f(\pi_G(q)) = \pi_{G^\prime}(\theta^*(q))$, so we have the following.
	\begin{align*}
		f^*(\pi_G(q)) & = f(\pi_G(q))                                                    \\
		              & = \pi_{G^\prime}(\theta^*(q))                                    \\
		              & = \pi_{G^\prime}(\theta^*(q_1\cdots q_n))                        \\
		              & = \pi_{G^\prime}(\theta(q_1) \cdots \theta(q_n))                 \\
		              & = \pi_{G^\prime}(\theta(q_1)) \cdots \pi_{G^\prime}(\theta(q_n)) \\
		              & = f(\pi_G(q_1)) \cdots f(\pi_G(q_n))                             \\
		              & = f^*(\pi_G(q_1)) \cdots f^*(\pi_G(q_n))
	\end{align*}
	Thus also
	\begin{align*}
		 & \pi_{G^\prime}(f^*(\pi_G(q)\pi_G(q_n)^{-1}\pi_G(q_{n-1})^{-1} \cdots \pi_G(q_1)^{-1})) =                   \\
		 & \pi_{G^\prime}(f^*(\pi_G(q_1))\cdots f^*(\pi_G(q_n))(f^*(\pi_G(q_n)))^{-1}\cdots (f^*(\pi_G(q_1)))^{-1}) = \\
		 & \pi_{G^\prime}(1)  = 1                                                                                     \\
	\end{align*}
\end{proof}

\begin{remark}
	Again, we note that the last part of the proof had no dependence on $G^\prime$.
	Once we had the map $f$, we just chased the diagram and everything popped out.
	This is because of the very specific form of the relations in  $G^Q$.
	I believe that  $G^Q$ is defined by the diagram in \cref{thm:commuting_diagram_GQ}, but I don't know enough category theory to explore that.
\end{remark}
\begin{remark}
	Note that in \cref{thm:commuting_diagram_GQ}, c.f.~\cref{lem:homomorphism_squares} we replaced $(S \cup S^{-1})^*$ with $Q$.
\end{remark}
\begin{remark}
	If we set $Q = S$ (the minimum language allowed), then  $G^Q$ is  $F_S$, the free group generated by  $S$.
	Then  $f$ always exists (and is  $\theta$) and \cref{thm:commuting_diagram_GQ} tells us the standard theorem about homomorphisms from the free group.
\end{remark}
\begin{remark}
	If we set $Q = (S \cup S^{-1})^*$ (the maximum language), then $G^Q$ is  $G$, $f$ is  $g$ and \cref{thm:commuting_diagram_GQ} tells us nothing more than  \cref{lem:homomorphism_squares}.
\end{remark}


