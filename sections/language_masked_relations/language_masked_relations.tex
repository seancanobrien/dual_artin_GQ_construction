%! TEX root = ../../main.tex
Let $F \colon \cset \to \cgrp$ be the free functor and denote its action on objects as $F_S \coloneqq F(S)$ and its action on morphisms as  $f^* \coloneqq F(f \colon S \to T)$. To each quotient $G$ of $F_S$ we associate the surjective homomorphism $\pi_{(S,G)} \colon F_S \to G$ which is projection. In this context, we can frame a basic group theoretic fact. We write $G \cong \GroupPres{S \relations R}$ if  $G \cong F_S / N(R)$ where $N(R)$ is the minimal normal subgroup in $F_S$ that contains $R$. We write  $G \cong \GroupPres{S}$ if  $G$ is isomorphic to some quotient of  $F_S$.

\begin{theorem}
	Suppose we have two groups $G_1$ and $G_2$. Suppose that $G_1 \cong \GroupPres{S \relations R}$ where $R$ is a subset of $F_S$ for which $\pi_{(S,G_1)}(R) = \Set{1}$.
	Given a map $f \colon S \to G_2$, if  $\pi_{(G_2, G_2)} \circ f^*(R) = \Set{1} \subseteq G_2$, then $f$ defines a homomorphism $h \colon G_2 \to G_2$.
	\label{thm:homomorphism_universal_prop}
\end{theorem}
\begin{proof}
	We have $G_1 \cong F_{S}/N(R)$.
	By assumption, $R \subseteq \ker(\pi_{(G_2,G_2)} \circ f^*)$, which is normal, thus $N(R) \subseteq \ker(\pi_{(G_2,G_2)} \circ f^*)$.
	By the universal property of the quotient we have the following commutative diagram where the unique map $h$ is the required homomorphism.
	\begin{equation*}
		\begin{tikzcd}
			F_{S} \ar[rr, "\pi_{(G_2,G_2)} \circ f^*"] \ar[d, "\pi_{(S,G_1)}"'] & & G_2 \\
			F_{S}/N(R) \cong G_1 \ar[rru, "\exists ! h"']
		\end{tikzcd}
	\end{equation*}
\end{proof}

The above theorem defines a homomorphism $h \colon \GroupPres{S \relations R} \cong G_1 \to G_2$ when a map $f \colon S \to G_2$ is compatible with the relations in $G_1$ and $G_2$.
If we also had knowledge of a generating set for $G_2$, we could construct homomorphisms in a different way.
\begin{theorem}
Suppose $G_1 \cong \GroupPres{S_1 \relations R_1}$ and $G_2 \cong \GroupPres{S_2}$.
Given a map $f \colon S_1 \to S_2$, if there exists a map  $h$ such that the following diagram commutes, then $h$ is a homomorphism.
\begin{equation}
\begin{tikzcd}
	S_1 \ar[d, "f"] \ar[r, hook, "i_{S_1}"] & F_{S_1} \ar[d, "f^*"] \ar[rr, "\pi_{(S_1, G_1)}"] & & \ar[d, color=red, "h"] G_1 \\
	S_2 \ar[r, hook, "i_{S_2}"] & F_{S_2} \ar[rr, "\pi_{(S_2, G_2)}"] & & G_2
\end{tikzcd}
\label{eqn:homomorphism_squares}
\end{equation}
	\label{thm:homomorphism_squares}
\end{theorem}
\begin{proof}
	Consider some $a,b \in G_1$. Since $\pi_{(S_1,G_1)}$ is surjective, we can choose some $\overline{a} \in \pi_{(S_1,G_1)}^{-1}(a)$ and $b \in \pi_{(S_1,G_1)}^{-1}(b)$.
	Since $\pi_{(S_1,G_1)}$ is a homomorphism, we have $\overline{a}\overline{b} \in \pi_{(S_1,G_1)}^{-1}(ab)$. So $h(ab) = h \circ \pi_{(S_1,G_1)}(\overline{a}\overline{b})$.
	By the commutative diagram, this is $\pi_{(S_2,G_2)} \circ f^* (\overline{a}\overline{b})$.
	Since $\pi_{(S_2,G_2)} \circ f^*$ is a homomorphism, this is
	\begin{align*}
		\pi_{(S_2,G_2)} \circ f^*\left(\overline{a}\right)\pi_{(S_2,G_2)} \circ f^*\left(\overline{b}\right) = 
		\\ h \circ \pi_{(S_1,G_1)}\left( \overline{a} \right) h \circ \pi_{(S_1,G_1)}\left( \overline{b} \right) =
		\\ h(a)h(b)
	\end{align*}
\end{proof}

\begin{corollary}
	A map $h \colon G_1 \to G_2$ is a homomorphism if and only if we can construct the relevant commuting diagram \eqref{eqn:homomorphism_squares}.
\end{corollary}
\begin{proof}
	\cref{thm:homomorphism_squares} gives us the $\impliedby$ direction. If we consider $G \cong \GroupPres{G_2}$ we can construct the  $\implies$ direction.
\end{proof}

\begin{remark}
	If we construct a homomorphism using \cref{thm:homomorphism_universal_prop}, then we get the middle commuting square of \eqref{eqn:homomorphism_squares} automatically.
	\label{rmk:universal_prop_homomorphism_gives_squares}
\end{remark}

Now consider \eqref{eqn:homomorphism_squares}, but replace $F_{S_1}$ with some subset $Q \subseteq F_{S_1}$. We will now construct a group, which we will denote $G^Q$, where such a diagram defines a homomorphism from $G^Q$ to $G_2$.

\begin{definition}[Group with relations visible in $Q$]
	 Suppose we have a group $G \cong \GroupPres{S}$.
	 Fix some $Q \subseteq F_S$ such that $S \subseteq Q$.
	 Let $\pi \coloneq \pi_{(S,G)}$.
	 We have the following maps.
	 \[
	 \begin{tikzcd}
		 S \ar[r, hook, "i"] & Q \ar[r, "\pi"] & \pi(Q)
	 \end{tikzcd}
	 .\]
	 Define the \emph{group $G$ with relations visible in $Q$}, to be  
	 \[
		 G^Q \coloneqq \GroupPres{\pi(Q) \relations \Set{\pi(q) = (\pi \circ i)^*(q) \given q \in Q} }
	 .\]
	 \label{def:G_Q}
\end{definition}

\begin{remark}
	Do not think of  the generators $\pi(Q)$ as elements of $G$.
	They are abstract generators.
	Specifically,  $G^Q$ is a quotient of  $F_{\pi(Q)}$.
	Thus, our relations should be equations in $F_{\pi(Q)}$, which they are.
\end{remark}

\begin{remark}
	$G^Q$ is generated by $\pi(S) \subseteq \pi(Q)$.
	$G$ is a quotient of  $G^Q$.
\end{remark}

\begin{lemma}
	Let $G \cong \GroupPres{S}$ be a group and consider the following setup.
	\[
	\begin{tikzcd}
		S \ar[r, hook, "i_S"] & F_S \ar[rr, "\pi_{(S,G)}"] & & G \ar[r, hook, "i_G"] & F_G \ar[rr, "\pi_{(G,G)}"] & & G
	\end{tikzcd}
	\]
	The following maps are equal.
	\[
		\pi_{(G,G)} \circ i_G \circ \pi_{(S,G)} = \pi_{(G,G)} \circ (\pi_{(S,G)} \circ i_S)^*
	.\]
	\label{lem:group_projections}
\end{lemma}

\begin{theorem}
	Suppose we have two groups $G_1 \cong \GroupPres{S_1}$ and $G_2 \cong \GroupPres{S_2}$.
	Fix some $Q \subseteq F_{S_1}$ such that  $S \subseteq Q$.
	Let $\overline{Q} \coloneqq \pi_{(S_1, G_1)}(Q) $.
	If in the following diagram \eqref{eqn:G_Q_homomorphism_setup}, if there exists a map $f$ that makes the diagram commute, then there is a homomorphism $h \colon (G_1)^Q \to G_2$.
	\begin{equation}
		\begin{tikzcd}
			S_1 \ar[d, "\theta"] \ar[r, hook, "i_{S_1}"] & Q \ar[d, "\theta^*"] \ar[rr, "\pi_{(S_1,G_1)}"] & & \overline{Q} \ar[d, color=red, "\exists f ?"] \\
			S_2 \ar[r, hook, "i_{S_2}"] & F_{S_2} \ar[rr, "\pi_{(S_2,G_2)}"] & & G_2 
		\end{tikzcd}
		\label{eqn:G_Q_homomorphism_setup}
	\end{equation}
	\label{thm:G_Q_homomorphism}
\end{theorem}

\begin{proof}
	Suppose we have such a map $f$.
	Note that  $\overline{Q}$ is a generating set for  $G^Q$, thus we have the setup of \cref{thm:homomorphism_universal_prop}.
	We use the following commutative diagram to define some inclusion maps and as a reference for the setup.
	\begin{equation}
		\begin{tikzcd}
			S_1 \ar[d, "\theta"] \ar[r, hook, "i_{S_1}"] & Q \ar[d, "\theta^*"] \ar[rr, "\pi_{(S_1, G_1)}"] & & \overline{Q} \ar[d, "f"] \ar[r, hook, "i_{\overline{Q}}"] & F_{\overline{Q}} \ar[d, "f^*"]
			\\ S_2 \ar[r, hook, "i_{S_2}"] & F_{S_2} \ar[rr, "\pi_{(S_2, G_2)}"] & & G_2 \ar[r, hook, "i_{G_2}"] & F_{G_2} \ar[rr, "\pi_{(G_2, G_2)}"] & & G_2
		\end{tikzcd}
		\label{eqn:G_Q_proof_setup}
	\end{equation}
	We can construct a homomorphism if $\pi_{(G_2, G_2)} \circ f^*(R) = \Set{1}$, where $R$ are the relations for $G^Q$, defined in \cref{def:G_Q}.
	For some $q \in Q$, the corresponding element in $R$ equalling the identity is  $i_{\overline{Q}} \circ \pi_{(S_1,G_1)}(q)(\pi_{(S_1,G_1)} \circ i_1)^*(q)^{-1}$.
	Using that $\pi_{(G_2,G_2)} \circ f^*$ is a homomorphism, we have
	\begin{align*}
		\pi_{(G_2,G_2)} \circ f^* \left(i_{\overline{Q}} \circ \pi_{(S_1,G_1)}(q)(\pi_{(S_1,G_1)} \circ i_1)^*(q)^{-1}\right) =
		\\ \left(\pi_{(G_2,G_2)} \circ f^* \circ i_{\overline{Q}} \circ \pi_{(S_1,G_1)} (q)\right) \left( \pi_{(G_2,G_2)} \circ f^* \circ\left( \pi_{(S_1,G_1)} \circ i_{S_1}\right)^*(q)^{-1} \right).
	\end{align*}
	We will concentrate on each factor separately.

	First we consider the first factor.
	Since $\pi_{(S_1, G_1)}(q) \in \overline{Q}$, by the rightmost commuting square of \eqref{eqn:G_Q_proof_setup}, we have $f^* \circ i_{\overline{Q}} \circ \pi_{(S_1, G_1)}(q) = i_{G_2} \circ f \circ \pi_{(S_1,G_1)} (q)$.
	This gives us
	\begin{align*}
		\pi_{(G_2,G_2)} \circ f^* \circ i_{\overline{Q}} \circ \pi_{(S_1,G_1)} (q) = \pi_{(G_2,G_2)} \circ i_{G_2} \circ f \circ \pi_{(S_1,G_1)}(q).
	\end{align*}
	We then use the middle commuting square of \eqref{eqn:G_Q_proof_setup} to give us
	\begin{align*}
		\pi_{(G_2,G_2)} \circ i_{G_2} \circ f \circ \pi_{(S_1,G_1)}(q) = \pi_{(G_2,G_2)} \circ i_{G_2} \circ \pi_{(S_2,G_2)} \circ \theta^*(q).
	\end{align*}
	We then use \cref{lem:group_projections} and functoriality, giving us
	\begin{align*}
		\pi_{(G_2,G_2)} \circ i_{G_2} \circ \pi_{(S_2,G_2)} \circ \theta^*(q) &= \pi_{(G_2,G_2)} \circ \left( \pi_{(S_2,G_2)} \circ i_{S_2} \right)^*  \circ \theta^*(q)
										   \\ &= \pi_{(G_2,G_2)} \circ \left( \pi_{(S_2,G_2)} \circ i_{S_2} \circ \theta \right)^*(q)
	\end{align*}

	We now concentrate on the second factor. Using functoriality, then the middle commuting square of \eqref{eqn:G_Q_proof_setup}, we get
	\begin{align*}
		\pi_{(G_2,G_2)} \circ f^* \circ\left( \pi_{(S_1,G_1)} \circ i_{S_1}\right)^*(q)^{-1} &= \pi_{(G_2,G_2)} \circ \left( f \circ \pi_{(S_1,G_1)} \circ i_{S_1}\right)^*(q)^{-1}
												  \\ &= \pi_{(G_2,G_2)} \circ \left(\pi_{(S_2,G_2)} \circ \theta^* \circ i_{S_1}\right)^*(q)^{-1}.
	\end{align*}
	Then, we use the leftmost commuting square of \eqref{eqn:G_Q_proof_setup} to get
	\begin{align*}
		\pi_{(G_2,G_2)} \circ \left(\pi_{(S_2,G_2)} \circ \theta^* \circ i_{S_1}\right)^*(q)^{-1} = \pi_{(G_2,G_2)} \circ \left(\pi_{(S_2,G_2)} \circ i_{S_2} \circ \theta \right)^*(q)^{-1}.
	\end{align*}

	This is the inverse of the left factor.
\end{proof}

\begin{remark}
	If we set $Q = S_1 \subseteq F_{S_1}$ (the minimum subset allowed), then  $G^Q$ is  $F_{S_1}$.
	In this case, $f$ always exists (and is  $\theta$) and \cref{thm:G_Q_homomorphism} tells us the standard theorem about homomorphisms from the free group.
\end{remark}

\begin{remark}
	If we set $Q = F_{S_1}$, then $G^Q$ is  $G$, $f$ is $g$ and \cref{thm:G_Q_homomorphism} tells us nothing more than  \cref{thm:homomorphism_squares}.
\end{remark}

\begin{corollary}
	The homomorphism $h$ resulting from \cref{thm:G_Q_homomorphism} makes the following diagram \eqref{eqn:G_Q_homomorphism_square} commute.
	Thus, considering $S_1$ as a generating set for  $(G_1)^Q$, $h$ is an extension of $\theta$, as in \cref{thm:homomorphism_squares}.
	\begin{equation}
		\begin{tikzcd}
			S_1 \ar[d, "\theta"] \ar[r, hook, "i_{S_1}"] & Q \ar[d, "\theta^*"] \ar[rr, "\pi_{(S_1, G_1)}"] & & \overline{Q} \ar[d, "f"] \ar[r, hook, "i_{\overline{Q}}"] & F_{\overline{Q}} \ar[d, "f^*"] \ar[rr, "\pi_{\left(\overline{Q}, (G_1)^Q\right)}"] & & (G_1)^Q \ar[d, "h"]
			\\ S_2 \ar[r, hook, "i_{S_2}"] & F_{S_2} \ar[rr, "\pi_{(S_2, G_2)}"] & & G_2 \ar[r, hook, "i_{G_2}"] & F_{G_2} \ar[rr, "\pi_{(G_2, G_2)}"] & & G_2
		\end{tikzcd}
		\label{eqn:G_Q_homomorphism_square}
	\end{equation}
\end{corollary}

\begin{proof}
	By \cref{rmk:universal_prop_homomorphism_gives_squares}, we get the rightmost square in \eqref{eqn:G_Q_homomorphism_square}.
\end{proof}

\begin{corollary}
	If $\theta \colon S_1 \to S_2$ is surjective, then the homomorphism $h \colon (G_1)^Q \to G_2$ resulting from \cref{thm:G_Q_homomorphism} is surjective.
\end{corollary}

We now consider altering $Q$.

\begin{corollary}
	Suppose we have some group $G \cong \GroupPres{S \relations R}$.
	Given some $Q,Q^\prime \subseteq F_{S}$ such that $S \subseteq Q \subseteq Q^\prime$, there exists a surjective homomorphism $a: G^{Q} \to G^{Q^\prime}$.
\end{corollary}

\begin{proof}
	Apply \cref{thm:G_Q_homomorphism} with $G_2 = G^{Q^\prime}$ and $\theta = \id_S$. We get the required homomorphism if there exists an $f$ such that the following diagram commutes.
	\begin{equation*}
		\begin{tikzcd}
			& Q \ar[dl, "\pi_{(S,G)}"] &
			\\ \overline{Q}
		\end{tikzcd}
	\end{equation*}
\end{proof}


