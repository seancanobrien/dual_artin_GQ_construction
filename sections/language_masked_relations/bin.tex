%! TEX root = ../../main.tex
We now consider altering $Q$.
If we increase $Q$ to some $Q^\prime$ where $Q \subseteq Q^\prime$, we see that both $G^Q$ and $G^{Q^\prime}$ are generated by the image of $S$, but the relations for $G^{Q^\prime}$ are a superset of the relations for $G^Q$.
Thus, $G^{Q^\prime}$ can be realised as a quotient of $G^Q$ via a surjective homomorphism $p$, for which the following diagram commutes.

\begin{equation}
	\begin{tikzcd}
		& F_S \ar[dl, "\pi_{\left( S, G^Q \right) }"'] \ar[dr, "\pi_{\left( S, G^{Q^\prime} \right) }"] &
		\\ G^Q \ar[rr, "p"] & & G^{Q^\prime}
	\end{tikzcd}
	\label{eqn:projecting_G_Q_to_G_Q_prime}
\end{equation}

Conversely, if $\id_S$ can extend to a homomorphism in the opposite direction $G^{Q^\prime} \to G^Q$, then $G^Q$ and $G^{Q^\prime}$ are isomorphic.
We can use \cref{thm:G_Q_homomorphism} to find when this is possible.

\begin{theorem}
	Given some group $G \cong \GroupPres{S}$, and $Q, Q^\prime \subseteq F_S$ such that $S \subseteq Q \subseteq Q^\prime$, if there exists a map $f$ that makes the following diagram commute, then $G^Q \cong G^{Q^\prime}$.

	\begin{equation*}
		\begin{tikzcd}
			& Q^\prime \ar[dl, "\pi_{\left( S, G \right) }"'] \ar[dr, "\pi_{\left( S, G^Q \right) }"] &
			\\ \pi_{\left( S, G \right) }(Q^\prime) \ar[rr, "f"] & & \pi_{\left( S,G^Q \right) }(Q^\prime)
		\end{tikzcd}
	\end{equation*}
	\label{thm:extending_Q_resulting_in_isomorphism}
\end{theorem}

\begin{proof}
	By \cref{thm:G_Q_homomorphism}, this we know that $\id_S$ extends to a homomorphism $a \colon G^{Q^\prime} \to G^Q$.
	Using $p \colon G^Q \to G^{Q^\prime}$ in \eqref{eqn:projecting_G_Q_to_G_Q_prime} gives us a homomorphism in the opposite direction such that $\left(p \circ a \right) |_S = \id_S$, thus $a$ is an isomorphism.
\end{proof}

A pleasant feature of \cref{thm:extending_Q_resulting_in_isomorphism} is that we only need to understand the behaviour of $\pi_{\left( S,G \right) }$ and $\pi_{\left( S, G^Q \right) }$ on $Q^\prime$ in order to know if we can increase $Q$ to $Q^\prime$.
We do not need to understand how $\pi_{\left(S,G^{Q^\prime}\right)}$ behaves.
This suggests the possibility of extending $Q$ inductively.

We have a well--defined notion of $Q$ being maximal.
If  $Q \subseteq F_S$ is such that any $Q^\prime \supset Q$ results in a $G^{Q^\prime} \ncong G^Q$, then we say $Q$ is maximal.
We also have an equivalent notion of $Q$ being minimal.

It is unclear whether a minimal or maximal $Q$ is unique.
