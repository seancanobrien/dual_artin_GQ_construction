%! TEX root = ../../main.tex
\begin{definition}
	\label{def:G_set}
	Let $G$ be a group and $X$ be a set.
	A \emph{$G$-set} is a set  $X$ with an accompanying  $G$ action, denoted  $g \cdot x$ for each  $g \in G$ and  $x \in X$.
\end{definition}
\begin{definition}
	Given a group $G$, suppose we have two $G$-sets,  $X$, and $X^\prime$.
	Now suppose we have a map  $f \colon X \to X^\prime$.
	This map is a \emph{$G$-set morphism} iff
	\[
		f(g \cdot x) = g \cdot f(x)
	\]
	for all $g \in G$ and  $x \in X$.
\end{definition}

Let  $B_n$ denote the braid group on  $n$ strands with standard generating set $\Set{\sigma_i}$.
\begin{definition}[The Hurwitz action]
	Given any group $G$, there is an action of  $B_n$ on  $G^n$, called the \emph{Hurwitz action} where
	\[
		\sigma_i \cdot  \left( g_1, \ldots g_n \right) = \left( g_1,\ldots,g_{i-1}, g_{i+1}^{\left(g_i^{-1}\right)}, g_{i}, g_{i+2}, \ldots, g_n \right)
	\]
	for all $1\leq i<n$.
	This also determines the action of $\sigma_i^{-1}$.
\end{definition}
This action preserves the index-wise product $(g_1, \ldots, g_n) \mapsto g_1g_2\cdots g_n$.
Given some $X \subseteq G$, if $X^G = G$, then the Hurwitz action is also well-defined on $X^n$, and this makes any such $X$ a  $B_n$-set in the sense of \cref{def:G_set}.
Given a map $f \colon X \to Y$, let  $f^{(n)}$ denote the map between $n$-tuples $(x_1,\ldots,x_n)\mapsto (f(x_1),\ldots,f(x_n))$.
If we set $\tau = \pi^{(n)}_W(s_1,\ldots,s_n)$, then by \cite{igusa_schiffler_exceptional_2010}, we know that $B_n \cdot \tau$ gives every minimal factorisation of  $w$ in to elements of $R$.
% In particular, we wish to consider this action on $S^n$, $T^n$ and $U^n$.

\begin{remark}
	\label{rmk:hurwitz_action_Bn_set_morphism}
	Given two groups $G_1$ and  $G_2$, and a homomorphism $\phi \colon G_1 \to G_2$, the map $\phi^{(n)} \colon G_1^n \to G_2^n$ is a  $B_n$-set morphism.
\end{remark}
Let $p_1 \colon G^n \to G$ denote generic projection to the first coordinate of a tuple.
We are interested in the tuples that occur due to the Hurwitz action.
The set of tuples $B_n \cdot \tau$ is the orbit of $\tau$.
The set $\hurref(\tau)$ is the set of all  $g \in G$ that occur in any position in  $B_n \cdot \tau$.
By choosing braids that move elements to the first position in the tuple, we see that $\hurref(\tau) = p_1(B_n \cdot \tau)$.

Associated to a tuple, there is a monoid construction, which we will call the \emph{Hurwitz monoid}.
\begin{definition}
	\label{def:hurwitz_group}
	Let $G$ be some group, and let $\tau$ be some tuple  in $G^k$.
	Suppose there in there is a tuple in  $B_k \cdot \tau$ that begins $r_1,r_2$.
	Let $s$ denote the element of $\hurref(\tau)$ equal in $G$ to  $r_1r_2r_1^{-1}$.
	We call any such $(r_1,r_2,s)$ a \emph{Hurwitz triple}.
	In this context, we define the \emph{Hurwitz monoid} $I$ associated to $\tau$ as follows.
	Decorate elements $x \in \hurref(\tau)$ as $[x]$ to distinguish them as generators of $I$ rather than elements of $G$.
	\[
		I \coloneq \GroupPres{[\hurref(\tau)] \relations \Set{ [r_1][r_2] = [s][r_1] \given (r_1,r_2,s) \text{ is a Hurwitz triple}}}
		.\]
\end{definition}
We call the corresponding group, the \emph{Hurwitz group}.
We denote the Hurwitz group associated to a tuple $\tau$ by  $H(\tau)$.

\begin{remark}
	The relations in the above definition are naturally associated to $\sigma_1$ acting on a tuple beginning with $r_1,r_2$.
	However, we can also get the relation associated to  $\sigma_1^{-1}$ acting on the same tuple.
	Suppose we have this tuple beginning with $r_1,r_2$ in $B_n \cdot \tau$.
	If we then act by  $\sigma_1^{-1}$, our tuple begins with $r_2,t$, where $t$ is the element in  $\hurref(\tau)$ that is equal to  $r_2^{-1}r_1r_2$ in $G$.
	The relation associated to this tuple beginning $r_2,t$ is
	\[
		[r_2][t]=[t^{-1}r_2t][r_2]
	\]
	where $t^{-1}r_2t=r_1$.
\end{remark}

\begin{lemma}
	\label{lem:inj_on_hurref_implies_iso_hurwitz_group}
	Suppose we have two groups $G_1$ and $G_2$ and a homomorphism $\phi \colon G_1 \to G_2$.
	Let $\tau_1 = (g_1,\ldots,g_n) \in G^n$ and $\tau_2 = \phi^n(\tau)$ be two tuples.
	If $\phi$ is injective on $\hurref(\tau_1)$, then  $H(\tau_1) \cong H(\tau_2)$.
\end{lemma}
\begin{proof}
	By \cref{rmk:hurwitz_action_Bn_set_morphism}, $\phi^n$ is a $B_n$-set morphism from  $B_n \cdot \tau_1$ to $B_n \cdot \tau_2$.
	If $\phi$ is injective on $\hurref(\tau_1)$, then $\phi^n$ is injective on  $B_n \cdot \tau$, so  $\phi^n|_{B_n\cdot \tau_1} \colon B_n \cdot \tau_1 \to B_n \cdot \tau_2$ is a  $B_n$-set isomorphism.
	Since $H(\tau_1)$ and  $H(\tau_2)$ are determined by the  $B_n$-sets  $B_n \cdot \tau_1$ and $B_n \cdot \tau_2$ respectively, the map $[x] \to [\phi(x)]$ for all $x \in \hurref(\tau_1)$ defines an isomorphism  $H(\tau_1) \to H(\tau_2)$.
\end{proof}

Let $\ell \colon B_n \to \N$ denote minimum braid length with respect to the standard generating set $\Set{\sigma_i} $.
We can associate a related length for elements of $\hurref(\tau)$.
\begin{definition}
	\label{def:hurref_braid_length}
	Given some tuple $\tau$, each $r \in \hurref(\tau)$ has an associated \emph{reflection braid length} defined as
	\[
		\ell_\tau(r) \coloneq \min\Set{\ell(\beta) \given r \in \beta \cdot \tau}
		.\]
\end{definition}

\begin{lemma}
	\label{lem:hurwitz_group_generators}
	Given a group $G$ and a tuple $\tau \in G^k$, we have that $H(\tau)$ is generated by $\Set{[\tau_i]}$.
\end{lemma}
\begin{proof}
	Let $x \in \hurref(\tau) \setminus \Set{\tau_i}$.
	Let $\beta \in B_n$ be a minimum length braid such that $\beta \cdot \tau =(\mu,\cdots,\mu_n)$ and $x=\mu_k$ for some $k$, i.e.~a $\beta$ such that $\ell(\beta) = \ell_\tau(r)$.
	We will show that such an $[x]$ can be written in terms of generators appearing in $\gamma \cdot \tau$ where  $\gamma$ is a braid with $\ell(\gamma)<\ell(\beta)$.
	Induction then completes the proof.

	Since $x \notin \Set{\tau_i}$, we know $\ell(\beta)>0$.
	Since $\beta$ is a braid of minimum length with respect to these generators, we know that there is a factorisation of $\beta$ in the $\Set{\sigma_i} $ such that the last factor is either $\sigma_k$ or  $\sigma_{k-1}^{-1}$, since these are the only generators whose action puts something new in the $k^{\text{th}}$ position of the tuple.

	Suppose the last factor of $\beta$ is $\sigma_k$.
	Let $(\nu_1,\ldots,\nu_n) = \beta\sigma_k^{-1} \cdot \tau$, so $[x] = [\mu_k] = [\nu_{k+1}\nu_k\nu_{k+1}^{-1}]$.
	Since $\ell(\beta \sigma_k^{-1}) < \ell(\beta)$, if we can express $x$ in terms of the set $\Set{[\nu_i]}$ then we would be done.
	Since we can bring $\nu_k$ and  $\nu_{k+1}$ to the start of the tuple, we have the equation $[x]=[\nu_{k+1}][\nu_k][\nu_{k+1}]^{-1}$ in $H(\tau)$.

	We make a similar argument if the last factor of $\beta$ is  $\sigma_{i-1}^{-1}$.
\end{proof}

The above lemma tells us that there is a presentation for any $H(\tau)$ in just the elements $\Set{[\tau_i]} $.
We now wish to develop this presentation for $H(\tau)$.

To every relation $[r_1][r_2]=[s][r_1]$ set out in \cref{def:hurwitz_group} we can associate a braid $\beta$ such that  $\beta \cdot \tau$ begins with $r_1,r_2$.
The relation shows how to express $[s]=[r_1r_2r_1^{-1}]$ in terms of $[r_1]$ and $[r_2]$.
The proof above shows us that we can express $[r_1]$ and $[r_2]$ in terms of elements in  $\hurref(\tau)$ which result from braids of lower length, and so on.
We can express this structure using a decorated binary tree in the following way

Suppose for now our tuple $\tau$ is $(a,b,c) \in G^3$.
We can act on this by $\sigma_2\sigma_1^{-1}$ to get $(b,b^{-1}abcb^{-1}a^{-1}b,b^{-1}ab)$.
Consider now the generator $[b^{-1}abcb^{-1}a^{-1}b]$, and the relation $[b^{-1}ab][c] = [b^{-1}abcb^{-1}a^{-1}b][b^{-1}ab]$ in $H(\tau)$.
Let us express this relation by the following tree, which should be understood as the encoding relation $[b^{-1}abcb^{-1}a^{-1}b]=[b^{-1}ab][c] [b^{-1}ab]^{-1}$.
\[
	\begin{tikzpicture}
		\def\vsize{1}
		\def\hsize{0.833}
		\tikzstyle{every label}=[font=\footnotesize]
		\tikzstyle{every node}=[font=\footnotesize]

		\node[FSCW, label=below:{$[b^{-1}abcb^{-1}a^{-1}b]$}] (top) at (0,0) {};
		\node[FSC, label=above:{$[b^{-1}ab]$}] (l) at ($ (top) + (-\hsize,\vsize) $) {};
		\node[FSC, label=above:{$[c]$}] (r) at ($ (top) + (\hsize,\vsize) $) {};

		\draw (l) to (top);
		\draw (r) to (top);
	\end{tikzpicture}
\]
The bottom node is white, which specifies that the last factor in the right-hand side of the relation is an inverse.
This corresponds to the positive braid $\sigma_2$ that \emph{created} the new generator $[b^{-1}abc b^{-1}a^{-1}b]$.
Note that the planar order in this picture ($[b^{-1}ab]$ on the left and $[c]$ on the right) is important.

Now consider the following tree, which should understood as encoding the relation $[b^{-1}ab]=[b]^{-1}[a][b]$.
\[
	\begin{tikzpicture}
		\def\vsize{1}
		\def\hsize{0.833}
		\tikzstyle{every label}=[font=\footnotesize]
		\tikzstyle{every node}=[font=\footnotesize]
		\node[FSCB, label=below:{$[b^{-1}ab]$}] (top) at (0,0) {};
		\node[FSC, label=above:{$[a]$}] (l) at ($ (top) + (-\hsize,\vsize) $) {};
		\node[FSC, label=above:{$[b]$}] (r) at ($ (top) + (\hsize,\vsize) $) {};

		\draw (l) to (top);
		\draw (r) to (top);
	\end{tikzpicture}
\]
The top node is black, which specifies that the first factor in the right-hand side of the relation is an inverse.
This corresponds to the negative braid $\sigma_1^{-1}$ that \emph{created} the new generator $[b^{-1}ab]$.

We can put these two trees together.
\begin{equation}
	\label{eqn:tree_example}
	\begin{tikzpicture}[baseline={(current bounding box.center)}]
		\def\vsize{1}
		\def\hsize{0.833}
		\tikzstyle{every label}=[font=\footnotesize]
		\tikzstyle{every node}=[font=\footnotesize]

		\node[FSCW, label=below:{$[b^{-1}abcb^{-1}a^{-1}b]$}] (top) at (0,0) {};
		\node[FSCB] (m) at ($ (top) + (-\hsize,\vsize) $) {};
		\node[FSC, label=above:{$[a]$}] (ll) at ($ (top) + (-2*\hsize,2*\vsize) $) {};
		\node[FSC, label=above:{$[b]$}] (lr) at ($ (top) + (0,2*\vsize) $) {};
		\node[FSC, label=above:{$[c]$}] (r) at ($ (top) + (2*\hsize,2*\vsize) $) {};

		\draw (l) to (top);
		\draw (ll) to (m);
		\draw (lr) to (m);
		\draw (r) to (top);
	\end{tikzpicture}
\end{equation}

This tree encodes the relation  $[b^{-1}abc b^{-1}a^{-1}b]=\left([b]^{-1}[a][b]\right)[c]\left([b]^{-1}[a][b]\right)^{-1}$, by parsing relations as we move up the tree.
Let us explicitly state how these trees relate to the objects we have already discussed.

Given some tuple $\tau$, each  $r \in \hurref(\tau)$ occurs in some position in  $\beta \cdot \tau$ for some braid  $\beta$.
This  $r$ was some conjugate of other elements of $\hurref(\tau)$, this gives us the bottom node and the two outgoing edges of the tree, and we continue in this fashion.
The proof of \cref{lem:hurwitz_group_generators} tells us that any $r \in \hurref(\tau)$, can be expressed using such a tree, where the top leaf nodes are labelled by elements of  $\Set{[\tau_i]}$.
The information that specifies the tree is the braid $\beta$ and which element (which index) of $\beta \cdot \tau$ is the relevant $r$.
Note that we do not assert any kind of uniqueness associated to this tree.
This is because we do not assert uniqueness of the braid $\beta$.

There is a pictorial (from the braid picture) way to draw these trees.
Suppose we are dealing with the braid $\sigma_2\sigma_1^{-1}$ as in the above example.
Since this is acting from the left, let us draw the reversed braid $\sigma_1^{-1}\sigma_2$, and highlight the relevant strand which has index 2 in $\sigma_2\sigma_1^{-1}\cdot\tau$ in blue.
\[
	\begin{tikzpicture}
		\pic[
		braid/strand 3/.style={blue},
		name=coordinates
		]{
		braid={s_1^{-1} s_2}
		};
		\node[at=(coordinates-1-s), label=above:{$[a]$}] {};
		\node[at=(coordinates-2-s), label=above:{$[b]$}] {};
		\node[at=(coordinates-3-s), label=above:{$[c]$}] {};
	\end{tikzpicture}
\]
For now, this blue strand is our \emph{strand of interest}.
Follow this strand of interest upwards until it passes under another strand or reaches the top of the braid.
We choose the colour of the node according to how the strand of interest passes under the other strand.
\[
	\scalebox{0.7}{
		\begin{tikzpicture}[baseline={(current bounding box.center)}]
			\pic[
				braid/strand 2/.style={blue},
				name=coordinates
			]{
				braid={s_1}
			};
		\end{tikzpicture}
	}
	\;\mathlarger{\mathlarger{\leadsto}}\;
	\begin{tikzpicture}
		\node[FSCW] {};
	\end{tikzpicture}
	\qquad
	\qquad
	\scalebox{0.7}{
		\begin{tikzpicture}[baseline={(current bounding box.center)}]
			\pic[
			braid/strand 1/.style={blue},
			name=coordinates
			]{
			braid={s_1^{-1}}
			};
		\end{tikzpicture}
	}
	\;\mathlarger{\mathlarger{\leadsto}}\;
	\begin{tikzpicture}
		\node[FSCB] {};
	\end{tikzpicture}
\]
Once we cross under and make this node in the tree, we follow the two strands that were involved in that crossing.
Separately, we consider each of these strands as our strands of interest and move up the tree generating the daughter trees from our initial node in the same fashion.

In our example, the initial blue strand crosses under strand 1, creating a white node.
One strand coming from that node goes to the top of the braid to the label $[c]$ without any (under) crossings.
The other strand does have an under crossing, which corresponds to the black node in \eqref{eqn:tree_example}.

We can also associate trees to relations coming from \cref{def:hurwitz_group}.
Here we must take note of \cref{rmk:relations_in_hurwitz_group}.
If a relation is naturally associated to a $\sigma_i^{-1}$ crossing, then the bottom node of our tree should be black, even though every relation in
We have a way to generate Given some relation $[r_1][r_2] = [s][r_1]$ as in \cref{def:hurwitz_group}, there is a braid $\beta$ such that the set
\[
	X = \Set{p_1(\beta \cdot \tau), p_2(\beta \cdot \tau), p_1(\sigma_1\beta\cdot \tau), p_2(\sigma_1\beta\cdot\tau)}
\]
contains all the generators present in that relation.
Choose an element $x \in X$ such that $\ell_\tau(x)$ is maximal.
Then the tree associated to that $x$ and whichever of $\beta\cdot \tau$ or $\sigma_1\beta\cdot\tau$ it occurs in encodes the relevant relation, as well as the relations of all the other reflections and relations used to get there.
This is the tree we associate to the relation $[r_1][r_2] = [s][r_1]$.
As with elements of $\hurref(\tau)$, we do not assert any kind of uniqueness associated to this tree.
This is because we do not assert uniqueness of the  braid chosen from $\beta$ or $\sigma_1\beta$.

However, note that not all trees correspond to a relation seen in \cref{def:hurwitz_group}.

\begin{lemma}
	\label{lem:hurwitz_group_alt_presentation}
	Let $G$ be a group.
	Let $\tau = (s_1,\ldots,s_n) \in F_S^n$ and let $\pi \colon F_S \to G$ denote a homomorphism.
	Let $i \colon S \hookrightarrow F_S$ denote the natural inclusion.
	The Hurwitz group associated to the tuple $\pi^{(n)}(\tau)$ is isomorphic to the following group presentation.
	\[
		J \coloneq \GroupPres{\pi(\hurref(\tau)) \relations \Set{\pi(q) = (\pi \circ i)_*(q) \given q \in \hurref(\tau)}}
		.\]
\end{lemma}
\begin{proof}
	Let $I$ denote the group presentation as defined in \cref{def:hurwitz_group}.
	The generating set of $I$ is exactly $[\hurref(\pi(\tau))]$.
	By \cref{rmk:hurwitz_action_Bn_set_morphism}, we have $\pi(\hurref(\tau)) = \hurref(\pi^{(n)}(\tau))$.
	Thus, the bracketing map $b \colon x \mapsto [x]$ is a bijection between  the generating sets of $J$ and $I$.
	Using this bijection $b$, we need to show that the relations of  $I$ are consequences of the relations of  $J$ and visa-versa.

	Let us begin by showing that the relations of  $J$ are consequences of the relations of  $I$.
	By the argument preceding this lemma, each $q \in \hurref(\tau)$, and each relation in $I$ corresponds to a tree.
	For each relation $R$ in $I$, we have a relation-tree $T(R)$.
	For each node in any such tree, we can take the subtree beneath that node, which also corresponds to a relation in $I$.
	For a node $n$ in $T(R)$, let  $R(n)$ denote the relation in $I$ corresponding to the subtree beneath $n$ in  $T(R)$.

	Let $q$ be some word in $\hurref(\tau)$.
	If $q \in S$, then the corresponding relation in $I$ is trivial, so we may assume  $q \notin S$.
	Let $T$ be a tree corresponding to $q$.
	Let $n$ be a node in $T$, and let $n_L$ and  $n_R$ be the two daughter nodes of $n$.
	These two daughter nodes correspond to words $q_L$ and  $q_R$ in $\hurref(\tau)$.
	We will show that if the relations corresponding to $q_L$ and  $q_R$ in $J$ are true in $I$, then the relation corresponding to $q$ is also true in $I$.
	This part of the proof will then follow using induction down the tree.
	The trivial relations $[s_i]=[s_i]$ corresponding to $q \in S \subseteq \hurref(\tau)$ form the base case for our induction.

	Recall  that $b$ is our bracketing map which is a bijection between generating sets.
	So, we have by assumption that  $[\pi(q_L)] = (b \circ \pi \circ i)(q_L)$ and $[\pi(q_R)] = (b \circ \pi \circ i)_*(q_R)$ in $I$.
	Assume the node at the top of the tree corresponding to $q$ is white.
	Thus, $q = q_Lq_Rq_L^{-1}$, and we have the relation  $[\pi(q)] = [\pi(q_L)][\pi(q_R)][\pi(q_L)]^{-1}$ in $I$.
	So in $I$ we have the following equation.
	\begin{align*}
		 & [\pi(q)]                                                                                  & = \\
		 & [\pi(q_L)][\pi(q_R)][\pi(q_L)]^{-1}                                                       & = \\
		 & (b \circ \pi \circ i)_*(q_L)(b \circ \pi \circ i)_*(q_R)(b \circ \pi \circ i)_*(q_L)^{-1} & = \\
		 & (b \circ \pi \circ i)_*(q_Lq_Rq_L^{-1})                                                   & = \\
		 & (b \circ \pi \circ i)_*(q).
	\end{align*}
	We make a similar argument if the node corresponding the word $q$ is black.
	This completes the proof that the relations in  $J$ are consequences of the relations in  $I$.

	Now we will show that the relations in $I$ are consequences of the relations in  $J$.
	Let $T$ now denote the tree corresponding to some relation in $I$.
\end{proof}


\begin{remark}
	\label{rmk:relations_in_hurwitz_group}
	The above lemma tells us that there is a presentation for any $H(\tau)$ in just the elements $\Set{[\tau_i]} $.
	When we transform the defining relations in to relations in the set $\Set{[\tau_i]} $, they have a very particular form.
	For example, considering the tuple $(a,b) \in G^2$, we have the relation
	\[
		[aba^{-1}]=[a][b][a]^{-1}
		.\]
	We see that this relation allows us to commute the square bracket decorations through the word that we get from the Hurwitz action, and every such defining relation can be re-written in this way.
	However, note that the data that defines the relation is the specific word $aba^{-1}$ that emerges by performing the Hurwitz action as if $a$ and $b$ are free group elements.
	If it was true that $aba^{-1} \stackrel{G}{=} xyz$, for some arbitrary $x,y,z \in G$, we would \emph{not} necessarily have the relation $[aba^{-1}]=[x][y][z]$, and there is no reason to assume that any of $[x]$, $[y]$ or $[z]$ are even generators in the defining presentation for $H(\tau)$.
	\cref{sec:G_Q} and \cref{lem:dual_artin_W_Q_isomorphism} expand on these observations.
\end{remark}

\begin{lemma}
	\label{lem:hurwitz_to_G_homomorphism}
	Given a group $G$ and a tuple $\tau = (g_1,\ldots,g_n)$, the map $[g_i] \mapsto g_i$ defines a surjection $H(\tau) \to \GroupPres{\Set{g_i}} \subseteq  G$.
\end{lemma}
\begin{proof}
	This is true because every relation of the form discussed in \cref{rmk:relations_in_hurwitz_group} is trivially true in  $G$.
\end{proof}

Let $a,b \in G$ for some group  $G$.
Let $\Pi_k(a,b)$ denote the alternating product of $a$ and  $b$, beginning with $a$, and of length  $k$.
For example  $\Pi_5(a,b)=ababa$.
\begin{lemma}
	\label{lem:artin_to_hurwitz_surjection}
	Let $G$ be a group such that there exists a homomorphism $\phi \colon A \to G$ from our Artin group $A$.
	Consider the tuple $\tau = (\phi \circ \pi_A)^n(s_1,\ldots,s_n) \in G^n$.
	The map $\pi_A(s_i) \mapsto [\phi\circ\pi_A(s_i)]$ defines a surjection from $A$ to $H(\tau)$.
\end{lemma}
\begin{proof}
	For each $i$, let $t_i$ denote  $\pi_A(s_i)$ and  $u_i$ denote  $\phi(t_i)$.
	We will show that the map $t_i \mapsto [u_i]$ extends to a homomorphism.
	To do so, we need to show that any Artin-like equation of the form $\Pi_k(t_i,t_j) = \Pi_k(t_j,t_j)$ associated to a defining relation in $A$ also holds in $H(\tau)$.

	For any $i < j$, there is a braid $\beta$ such that  $\beta \cdot (u_1,\ldots,u_n)$ begins with $u_i,u_j$.
	Thus, without loss of generality, we can assume for the following argument that $i=1$ and  $j=2$.
	We only care about what happens at the first two places of this tuple, so we will only consider the Hurwitz action on the duple $(u_1,u_2)$.
	Consider the function $f_n(a,b) \coloneq \Pi_{n+1}(a,b)\Pi_n(a,b)^{-1} $.
	One can compute that $f_n(a,b)^{f_{n-1}(a,b)^{-1}}=f_{n+1}(a,b)$.
	Acting once by $\sigma_1$ on our duple we have
	\[
		\sigma_1 \cdot (u_1,u_2) = (u_1u_2u_1^{-1},u_1) = (f_1(u_1,u_2),f_0(u_1,u_2))
		.\]
	Thus, for $k>0$, we have
	\[
		\sigma_1^k \cdot (u_1,u_2) = (f_{k}(u_1,u_2),f_{k-1}(u_1,u_2))
		.\]
	So for each $k \geq 0$, we have a generator $[f_k(u_1,u_2)]$ in $H$, and as in \cref{rmk:relations_in_hurwitz_group} the corresponding relation
	\[
		[f_k(u_1,u_2)]=f_k([u_1],[u_2])
		.\]
	Suppose that $A$ has the relation  $\Pi_k(t_1,t_2)=\Pi_k(t_2,t_1)$, then in $H$ we have
	\begin{align*}
		f_k([u_1],[u_2])                               & = [f_k(u_1,u_2)] =  [\phi(f_k(t_1,t_2))] =                     \\
		[\phi(\Pi_{k+1}(t_1,t_2)\Pi_{k}(t_1,t_2)^{-1}] & = [\phi\left(\Pi_{k+1}(t_1,t_2)\Pi_{k}(t_2,t_1)^{-1}\right)] = \\
		[\phi(t_1)]                                    & = [u_1]
		.\end{align*}
	So we have the relation $\Pi_{k}([u_1],[u_2]) = \Pi_k([u_2],[u_1])$ in $H(\tau)$ as required.
	This map is surjective by \cref{lem:hurwitz_group_generators}.
\end{proof}

\begin{corollary}
	\label{cor:A_iso_to_hurwitz_A}
	Let $\tau = \pi_A^n(s_1,\ldots,s_n)$.
	We have that $H(\tau) \cong A$.
\end{corollary}
\begin{proof}
	\cref{lem:artin_to_hurwitz_surjection} gives a homomorphism $a \colon A \to H(\tau)$, and \cref{lem:hurwitz_to_G_homomorphism} gives a homomorphism $b \colon H(\tau) \to A$.
	The composition $b \circ a$ is the identity on a generating set for $A$.
\end{proof}

The following is not a standard definition, but it is equivalent to the standard definition by \cite[Lemma 7.11]{bessis_topology_2004}.
\begin{definition}
	Given a Coxeter element $w \in W$, the \emph{dual Artin group associated to  $w$}, denoted $A^\vee_w$, is defined to be the Hurwitz group associated to any $(r_1,\ldots,r_n)$ such that all $r_i \in R$ and  $r_1r_2\cdots r_n=w$.
\end{definition}
The obvious tuple to consider is $(\pi_W(s_1),\ldots,\pi_W(s_n))$.
Since our choice of $w$ is constant, we will denote our dual Artin group  $A^\vee$ (omitting any mention of  $w$).

\begin{corollary}
	\label{cor:surj_dual_artin_to_W}
	The natural map $\psi \colon [S_W] \to W$ which acts as  $[\pi_W(s_i)] \mapsto \pi_W(s_i)$ defines a surjective homomorphism $A^\vee \to W$.
\end{corollary}
\begin{proof}
	This directly follows from \cref{lem:hurwitz_to_G_homomorphism}.
\end{proof}

\begin{corollary}
	\label{cor:surj_artin_to_dual_artin}
	The map natural map $\phi \colon S_A \to A^\vee $ which acts as $\pi_A(s_i) \mapsto [q \circ \pi_A(s_i)] = [\pi_W(s_i)]$ extends to a surjective homomorphism $\phi \colon A \to A^\vee$.
\end{corollary}
\begin{proof}
	This directly follows from \cref{lem:artin_to_hurwitz_surjection}.
\end{proof}

\begin{theorem}[\hspace{1sp}{\cite{bessis_dual_2006}}]
	\label{thm:proj_inj_on_Q_implies_dual_isomorphism}
	Let $\tau = \pi^n_A(s_1,\ldots,s_n)$.
	If the natural surjection $q \colon A \to W$ is injective on  $\hurref(\tau)$, then $A^\vee \cong A$.
\end{theorem}
\begin{proof}
	By \cref{cor:A_iso_to_hurwitz_A}, we have $A \cong H(\tau)$, and by the hypothesis and \cref{lem:inj_on_hurref_implies_iso_hurwitz_group} we have $A^\vee \cong H(\tau)$.
\end{proof}


