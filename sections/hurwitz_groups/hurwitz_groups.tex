%! TEX root = ../../main.tex
Let  $B_n$ denote the braid group on  $n$ strands with standard generating set $\Set{\sigma_i}_{1 \leq i < n} $.
\begin{definition}[The Hurwitz action]
	Given any group $G$, there is an action of  $B_n$ on  $G^n$, called the \emph{Hurwitz action} where
	\[
		\sigma_i \cdot  \left( g_1, \ldots g_n \right) = \left( g_1,\ldots,g_{i-1}, g_{i+1}^{\left(g_i^{-1}\right)}, g_{i}, g_{i+2}, \ldots, g_n \right)
	\]
	for all $1\leq i<n$.
	This also determines the action of $\sigma_i^{-1}$.
\end{definition}
This action preserves the index-wise product $(g_1, \ldots, g_n) \mapsto g_1g_2\cdots g_n$.
Given some $X \subseteq G$, if $X^G = G$, then the Hurwitz action is also well-defined on $X^n$, and this makes any such $X$ a  $B_n$-set.
If we set $\tau = \pi^n_W(s_1,\ldots,s_n)$, then by \cite{igusa_schiffler_exceptional_2010}, we know that $B_n \cdot \tau$ gives every minimal factorisation of  $w$ in to elements of $R$.
% In particular, we wish to consider this action on $S^n$, $T^n$ and $U^n$.

\begin{remark}
	\label{rmk:hurwitz_action_Bn_set_morphism}
	Given two groups $G_1$ and  $G_2$, and a homomorphism $\phi \colon G_1 \to G_2$, the map $\phi^n \colon G_1^n \to G_2^n$ is a  $B_n$-set morphism.
\end{remark}
Let $p_1 \colon G^n \to G$ denote generic projection to the first coordinate of a tuple.
We are interested in the tuples that occur due to the Hurwitz action.
The set $B_n \cdot \tau$ is the set of all such tuples for some tuple  $\tau$.
The set $\hurref(\tau)$ is the set of all  $g \in G$ that occur in any position in  $B_n \cdot \tau$.
By choosing braids that move elements to the first position in the tuple, we see that $\hurref(\tau) = p_1(B_n \cdot \tau)$.

Associated to a tuple, there is a monoid construction, which we will call the \emph{Hurwitz monoid}.
\begin{definition}
	Let $G$ be some group, and let $\tau$ be some tuple  in $G^k$.
	Let $X$ denote  $\hurref(\tau)$.
	Suppose there in there is a tuple in  $B_k \cdot \tau$ that begins $r_1,r_2$.
	Let $r_3$ denote the element of $X$ equal in $G$ to  $r_1^{r_2^{-1}}$.
	We call any such $(r_1,r_2,r_3)$ a \emph{Hurwitz triple}.
	In this context, we define the \emph{Hurwitz monoid} $I$ associated to $\tau$ as follows.
	Decorate elements $x \in X$ as $[x]$ to distinguish them as generators of $I$ rather than elements of $G$.
	\[
		I \coloneq \GroupPres{[X] \relations \Set{ [r_1][r_2] = [r_3][r_1] \given (r_1,r_2,r_3) \text{ is a Hurwitz triple}}}
		.\]
\end{definition}
We call the corresponding group, the \emph{Hurwitz group}.
We denote the Hurwitz group associated to a tuple $\tau$ by  $H(\tau)$.

\begin{lemma}
	\label{lem:inj_on_hurref_implies_iso_hurwitz_group}
	Suppose we have two groups $G_1$ and $G_2$ and a homomorphism $\phi \colon G_1 \to G_2$.
	Let $\tau_1 = (g_1,\ldots,g_n) \in G^n$ and $\tau_2 = \phi^n(\tau)$ be two tuples.
	If $\phi$ is injective on $\hurref(\tau_1)$, then  $H(\tau_1) \cong H(\tau_2)$.
\end{lemma}
\begin{proof}
	By \cref{rmk:hurwitz_action_Bn_set_morphism}, $\phi^n$ is a $B_n$-set morphism from  $B_n \cdot \tau_1$ to $B_n \cdot \tau_2$.
	If $\phi$ is injective on $\hurref(\tau_1)$, then $\phi^n$ is injective on  $B_n \cdot \tau$, so  $\phi^n|_{B_n\cdot \tau_1} \colon B_n \cdot \tau_1 \to B_n \cdot \tau_2$ is a  $B_n$-set isomorphism.
	Since $H(\tau_1)$ and  $H(\tau_2)$ are determined by the  $B_n$-sets  $B_n \cdot \tau_1$ and $B_n \cdot \tau_2$ respectively, the map $[x] \to [\phi(x)]$ for all $x \in \hurref(\tau_1)$ defines an isomorphism  $H(\tau_1) \to H(\tau_2)$.
\end{proof}

\begin{lemma}
	\label{lem:hurwitz_group_generators}
	Given a group $G$ and a tuple $\tau \in G^k$, we have that $H(\tau)$ is generated by $\Set{\tau_i}$.
\end{lemma}
\begin{proof}
	Let $\ell \colon B_n \to \N$ denote minimum word length with respect to the standard generating set for $B_n$.
	Let $x \in \hurref(\tau) \setminus \Set{\tau_i}$.
	Let $\beta \in B_n$ be the minimum length braid such that $\beta \cdot \tau =(\mu,\cdots,\mu_n)$ and $x=\mu_i$ for some $i$.
	We will show that such an $[x]$ can be written in terms of generators appearing in $\gamma \cdot \tau$ where  $\gamma$ is a braid with $\ell(\gamma)<\ell(\beta)$.
	Induction then completes the proof.

	Since $x \notin \Set{\tau_j}$, we know $\ell(\beta)>0$.
	Recall our notation for the set of standard generators for $B_n$ is $\Set{\sigma_j}$.
	Since $\beta$ is a braid of minimum length with respect to these generators, we know that there is a factorisation of $\beta$ in the $\Set{\sigma_j} $ such that the last factor is either $\sigma_i$ or  $\sigma_{i-1}^{-1}$, since these are the only generators whose action puts something new in the $i^{\text{th}}$ position of the tuple.

	Suppose the last factor of $\beta$ is $\sigma_i$.
	Let $(\nu_1,\ldots,\nu_n) = \beta\sigma_i^{-1} \cdot \tau$, so $[x] = [\mu_i] = [\nu_{i+1}\nu_i\nu_{i+1}^{-1}]$.
	Since $\ell(\beta \sigma_i^{-1}) \leq \ell(\beta)-1$, if we can express $x$ in terms of the set $\Set{[\nu_j]}$ then we would be done.
	Since we can bring $\nu_i$ and  $\nu_{i+1}$ to the start of the tuple, we have the equation $[x]=[\nu_{i+1}][\nu_i][\nu_{i+1}]^{-1}$ in $H(\tau)$.

	We make a similar argument if the last factor of $\beta$ is  $\sigma_{i-1}^{-1}$.
\end{proof}

\begin{remark}
	\label{rmk:relations_in_hurwitz_group}
	The above lemma tells us that there is a presentation for any $H(\tau)$ in just the elements $\Set{[\tau_i]} $.
	When we transform the defining relations in to relations in the set $\Set{[\tau_i]} $, they have a very particular form.
	For example, considering the tuple $(a,b) \in G^2$, we have the relation
	\[
		[aba^{-1}]=[a][b][a]^{-1}
		.\]
	We see that this relation allows us to commute the square bracket decorations through the word that we get from the Hurwitz action, and every such defining relation can be re-written in this way.
	However, note that the data that defines the relation is the specific word $aba^{-1}$ that emerges by performing the Hurwitz action as if $a$ and $b$ are free group elements.
	If it was true that $aba^{-1} \stackrel{G}{=} xyz$, for some arbitrary $x,y,z \in G$, we would \emph{not} necessarily have the relation $[aba^{-1}]=[x][y][z]$, and there is no reason to assume that any of $[x]$, $[y]$ or $[z]$ are even generators in the defining presentation for $H(\tau)$.
\end{remark}

\begin{lemma}
	\label{lem:hurwitz_to_G_homomorphism}
	Given a group $G$ and a tuple $\tau = (g_1,\ldots,g_n)$, the map $[g_i] \mapsto g_i$ defines a surjection $H(\tau) \to \GroupPres{\Set{g_i}} \subseteq  G$.
\end{lemma}
\begin{proof}
	This is true because every relation of the form discussed in \cref{rmk:relations_in_hurwitz_group} is trivially true in  $G$.
\end{proof}

Let $a,b \in G$ for some group  $G$.
Let $\Pi_k(a,b)$ denote the alternating product of $a$ and  $b$, beginning with $a$, and of length  $k$.
For example  $\Pi_5(a,b)=ababa$.
\begin{lemma}
	\label{lem:artin_to_hurwitz_surjection}
	Let $G$ be a group such that there exists a homomorphism $\phi \colon A \to G$ from our Artin group $A$.
	Consider the tuple $\tau = (\phi \circ \pi_A)^n(s_1,\ldots,s_n) \in G^n$.
	The map $\pi_A(s_i) \mapsto [\phi\circ\pi_A(s_i)]$ defines a surjection from $A$ to $H(\tau)$.
\end{lemma}
\begin{proof}
	For each $i$, let $t_i$ denote  $\pi_A(s_i)$ and  $u_i$ denote  $\phi(t_i)$.
	We will show that the map $t_i \mapsto [u_i]$ extends to a homomorphism.
	To do so, we need to show that any Artin-like equation of the form $\Pi_k(t_i,t_j) = \Pi_k(t_j,t_j)$ associated to a defining relation in $A$ also holds in $H(\tau)$.

	For any $i < j$, there is a braid $\beta$ such that  $\beta \cdot (u_1,\ldots,u_n)$ begins with $u_i,u_j$.
	Thus, without loss of generality, we can assume for the following argument that $i=1$ and  $j=2$.
	We only care about what happens at the first two places of this tuple, so we will only consider the Hurwitz action on the duple $(u_1,u_2)$.
	Consider the function $f_n(a,b) \coloneq \Pi_{n+1}(a,b)\Pi_n(a,b)^{-1} $.
	One can compute that $f_n(a,b)^{f_{n-1}(a,b)^{-1}}=f_{n+1}(a,b)$.
	Acting once by $\sigma_1$ on our duple we have
	\[
		\sigma_1 \cdot (u_1,u_2) = (u_1u_2u_1^{-1},u_1) = (f_1(u_1,u_2),f_0(u_1,u_2))
		.\]
	Thus, for $k>0$, we have
	\[
		\sigma_1^k \cdot (u_1,u_2) = (f_{k}(u_1,u_2),f_{k-1}(u_1,u_2))
		.\]
	So for each $k \geq 0$, we have a generator $[f_k(u_1,u_2)]$ in $H$, and as in \cref{rmk:relations_in_hurwitz_group} the corresponding relation
	\[
		[f_k(u_1,u_2)]=f_k([u_1],[u_2])
		.\]
	Suppose that $A$ has the relation  $\Pi_k(t_1,t_2)=\Pi_k(t_2,t_1)$, then in $H$ we have
	\begin{align*}
		f_k([u_1],[u_2])                               & = [f_k(u_1,u_2)] =  [\phi(f_k(t_1,t_2))] =                     \\
		[\phi(\Pi_{k+1}(t_1,t_2)\Pi_{k}(t_1,t_2)^{-1}] & = [\phi\left(\Pi_{k+1}(t_1,t_2)\Pi_{k}(t_2,t_1)^{-1}\right)] = \\
		[\phi(t_1)]                                    & = [u_1]
		.\end{align*}
	So we have the relation $\Pi_{k}([u_1],[u_2]) = \Pi_k([u_2],[u_1])$ in $H(\tau)$ as required.
	This map is surjective by \cref{lem:hurwitz_group_generators}.
\end{proof}

\begin{corollary}
	Let $\tau = \pi_A^n(s_1,\ldots,s_n)$.
	We have that $H(\tau) \cong A$.
\end{corollary}
\begin{proof}
	\cref{lem:artin_to_hurwitz_surjection} gives a homomorphism $a \colon A \to H(\tau)$, and \cref{lem:hurwitz_to_G_homomorphism} gives a homomorphism $b \colon H(\tau) \to A$.
	The composition $b \circ a$ is the identity on a generating set for $A$.
\end{proof}

The following is not a standard definition, but it is equivalent to the standard definition by \cite[Lemma 7.11]{bessis_topology_2004}.
\begin{definition}
	Given a Coxeter element $w \in W$, the \emph{dual Artin group associated to  $w$}, denoted $A^\vee_w$, is defined to be the Hurwitz group associated to any $(r_1,\ldots,r_n)$ such that all $r_i \in R$ and  $r_1r_2\cdots r_n=w$.
\end{definition}
The obvious tuple to consider is $(\pi_W(s_1),\ldots,\pi_W(s_n))$.
Since our choice of $w$ is constant, we will denote our dual Artin group  $A^\vee$ (omitting any mention of  $w$).

\begin{theorem}[\hspace{1sp}{\cite{bessis_dual_2006}}]
	Let $\tau = \pi^n_A(s_1,\ldots,s_n)$.
	If the standard surjection $q \colon A \to W$ is injective on  $\hurref(\tau)$, then $A^\vee \cong A$.
\end{theorem}
\begin{proof}
	By the corollary above, $A \cong H(\tau)$, and by the hypothesis and \cref{lem:inj_on_hurref_implies_iso_hurwitz_group} we have $A^\vee \cong H(\tau)$.
\end{proof}


