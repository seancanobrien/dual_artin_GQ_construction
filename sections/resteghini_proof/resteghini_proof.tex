We will overview a proof in \cite{resteghini_free_2024} while trying to avoid unnecessary detail.
The generators $S = \Set{s_1, \ldots , s_n} $ have a natural bijection with generators of $A$.
Let $\overline{s_i}$ denote the generator in $A$ corresponding to $s_i \in S \subseteq W$.
We have an action of $B_n$ on $A^n$ defined in the exact same way as in \cref{def:braid_action}.
We will consider the orbit of $\overline{h} = (\overline{s_1}, \ldots, \overline{s_n})$ under this action.

\begin{definition}
	Let $G$ be a group.
	Let $\rho_n \colon G^n \to G$ be projection on to coordinate $n$.
	We define the map ${-} \codot {-} \colon B_n \times G^n \to G$ as follows
	\[
		\tau \codot (g_1, \ldots, g_n) = \rho_n(\tau \cdot (g_1, \ldots, g_n))
		.\]
\end{definition}

\begin{remark}
	$G^n \neq G$ so this is not a group action!
\end{remark}

\begin{definition}
	Define the equivalence relation $\sim$ on $B_n$ by
	\[
		\tau \sim \tau^\prime \iff \tau \codot h = \tau^\prime \codot h
		.\]
\end{definition}

\begin{definition}
	Define the equivalence relation $\dsim$ on $B_n$ by
	\[
		\tau \dsim \tau^\prime \iff \tau \codot \overline{h} = \tau^\prime \codot \overline{h}
		.\]
\end{definition}

\begin{theorem}[\hspace{1sp}{\cite[Propsition 3.9]{resteghini_free_2024}}]
	The $B_n$ relations $\sim$ and $\dsim$ are the same iff $A^\vee$ is isomorphic to $A$.
	\label{thm:resteghini_proof}
\end{theorem}
\begin{proof}
	We will only concentrate on the $\implies$ direction.
Furthermore, it is easy enough to see that $\tau \dsim \tau^\prime \implies \tau \sim \tau^\prime$.
	As such, we will only prove that if $\tau \sim \tau^\prime \implies \tau \dsim \tau^\prime$ for all $\tau,\tau^\prime \in B_n$, then $A\cong A^\vee$.

	Define $\psi \colon T_0 \to A$ in the following way.
	For $t \in T_0$, there exists a $\tau \in B_n$ such that $\tau \codot h = t$.
	Define  $\psi(t) = \tau \codot \overline{h}$.
	We will show this does not depend on our choice $\tau$.
	Suppose that $\tau^\prime \codot h$ is also $t$.
	Then  $\tau \sim \tau^\prime$.
	Then, by assumption $\tau \dsim \tau^\prime$, so $\tau^\prime\codot \overline{h} = \tau \codot\overline{h}$.

	So we have a map from the generating set of $A^\vee$ to $A$.
	Recall the presentation of $A^\vee$ from \cref{thm:dual_braid_relations}.
	Let $R$ be the set of relations (words equal to the identity) from that presentation.
	As in \cref{lem:extend_map_to_homomorphism}, if we can show that the natural extension of $\psi$ to $R$, maps all of $R$ to the identity in $A$, then we can extend $\psi$ to a homomorphism $\psi \colon A^\vee \to A$.

	Let $t$, $t^\prime$ and $t t^\prime t$ be as in \cref{thm:dual_braid_relations}.
	Let $\tau B_n$ be such that
	\[
		\tau \cdot h = (x_1, x_2, \ldots, x_{n-2}, t, t^\prime)
		.\]
	Let $\tau \cdot \overline{h} = (z_1, \ldots z_n)$.
	Thus, we have the following.
	\begin{align*}
		\psi(t^\prime)     & = \tau \cdot \overline{h} = z_n                                    \\
		\psi(t)            & = (\sigma_{n-1}\tau) \cdot \overline{h} = z_{n-1}                  \\
		\psi(t t^\prime t) & = (\sigma_{n-1}^2\tau) \cdot \overline{h} = z_{n-1}z_nz_{n-1}^{-1} \\
	\end{align*}

	Thus, $\psi( [t t^\prime t][t][t^\prime]^{-1}[t]^{-1} ) = 1$ and $\psi$ extends to a homomorphism $\psi \colon A^\vee \to A$.

	Let $\phi$ denote the homomorphism in \cref{thm:homo_art_to_dual_art}.
	Note that $\psi \circ \phi |_{\overline{S}} = \id_{\overline{S}}$, thus  $\psi \circ \phi$ is the identity $\psi$ is an isomorphism.
\end{proof}

This proof gives the distinct impression of ``what just happened''.
Most striking, is how in showing $\psi$ was a homomorphism, at no point did anything to do with $A$ play a part.
All we needed was that $A$ was a group, and it all popped out.
In fact, the key step was the existence of a map $\psi$, we got homomorphism for free.
This is due to the structure of the relations in \cref{thm:dual_braid_relations}.
We wish to now abstract the construction of the dual braid group emerging in \cref{thm:dual_braid_relations} and develop that in to a reconstruction of the above proof.


